\documentclass[a4paper, 11pt]{article}
\setlength{\topmargin}{-.5in}
\setlength{\textheight}{9in}
\setlength{\oddsidemargin}{.100in}
\setlength{\textwidth}{6.25in}
\usepackage [dutch] {babel}
\usepackage{graphicx}
\usepackage{amsmath}
\usepackage{fullpage}
\usepackage{listings}
\usepackage{color}
\usepackage{soul}
\usepackage{tikz}
\usepackage{hyperref}
\usepackage{parskip}

\definecolor{dkgreen}{rgb}{0,0.6,0}
\definecolor{gray}{rgb}{0.5,0.5,0.5}
\definecolor{mauve}{rgb}{0.58,0,0.82}

\lstset{frame=tb,
  language=Java,
  aboveskip=3mm,
  belowskip=3mm,
  showstringspaces=false,
  columns=flexible,
  basicstyle={\small\ttfamily},
  numbers=none,
  numberstyle=\tiny\color{gray},
  keywordstyle=\color{blue},
  commentstyle=\color{dkgreen},
  stringstyle=\color{mauve},
  breaklines=true,
  breakatwhitespace=true
  tabsize=3
}

\newcommand{\todo}[1]{{\huge \textcolor{green}{#1}}\\}
\newcommand{\comment}[1]{\textcolor{red}{#1}\\}
\newcommand{\viscomment}[1]{\textcolor{red}{#1}}
\newcommand{\flip}[1] {\textcolor{black}{#1}}

\begin{document}

\begin{titlepage}
\title{Thesis Follow Up}
\author{Sophie Marien}
\date{} 
\end{titlepage}



\maketitle


Thesis follow up vragen van Maart en April terug beantwoord. 
\section{Hoever sta je? Moet het werkplan aangepast worden?}
Het werkplan werd aangepast in functie van indiening in Augustus. Doelstellingen voor juni zijn: afwerken van de tekst van hoofdstukken 1,2, 3 en 5 en finaliseren van de formules voor de Nash evenwichten.  Momenteel moet ik nog de tekst verfijnen van hoofdstuk 5 over worm propagatie en de related work. De formules voor de nash equilibria zijn al deels ontwikkelde, maar moeten nog verder uitgewerkt worden. De week van 22 juni is gereserveerd voor de examens. Vanaf 27 juni ga ik terug verder werken aan het berekenen van de Nash equilibria.  Tussen 6 en 31 juli wordt de tekst door vrienden nagelezen op tikfouten, structuur en verstaanbaarheid. In augustus wordt de tekst van hoofdstukken 4 en 6 afgewerkt 
\section{Wat zijn de belangrijkste elementen uit het besluit dat je gaat schrijven in je tekst?}
\begin{itemize}
\item vaststellen wat de nash equilibria zijn
\item conclusies voor de verdedigingsstrategie"en van de defender
\item conclusies voor de aanvalsstrategie"en van de attacker
\item Ide"en voor verder onderzoek. 
\end{itemize}

\section{Verfijn de outline van je thesistekst. Geef per hoofdstuk de subsecties, en per subsectie het kernidee en een schatting van het aantal bladzijdes van die subsectie.}

\begin{enumerate}
\item \textbf{Introduction}(4 blz., final draft afgewerkt) 
\begin{itemize}
\item Research questions
\item Contributions and results
\item Conjunctures and open problems
\item Overview thesis
\end{itemize}
\item \textbf{General context}(11 blz., final draft afgewerkt)
\begin{itemize}
\item[2.1] What is security
\item[2.2] A brief introduction to game theory
\item[2.3] The FlipIt game
\item[2.4] Related Work on Extensions to FlipIt
\end{itemize}
\item \textbf{FlipIt with propagation delay} (13 blz., final draft afgewerkt)
\begin{itemize}
\item[3.1] Introduction
\item[3.2] Difference between FlipIt with and without propagation delay
\begin{itemize}
\item Single resource
\item Only one action per player
\item Immediate effect of the move
\item Stealth character of the moves
\item Cost associated with the move
\item Strategies
\item Application: Wormlike ATP
\end{itemize}
\item[3.3] Formalization of the periodic game with propagation delay
\end{itemize}
\item \textbf{Nash Equilibria}(formules deels uitgewerkt, geschatte lenge: 12 blz.)
\begin{itemize}
\item[4.1] Determining piecewise functions $opt_{D}(\delta_{A})$
\item[4.2] Determining piecewise functions $opt_{A}(\delta_{D})$
\item[4.3] Determining the Nash equilibria
\end{itemize}
\item\textbf{Models for the Delay}(8 blz., final draft afgewerkt)
\begin{itemize}
\item[5.1] Methods of worm propagation
\item[5.2] Graph models
\item[5.3] Models for worm propagation
\item[5.4] Method for calculating propagation delay through matrices
\end{itemize}
\item \textbf{Related Work: Alternative approaches to the security problem}  
\begin{itemize}
\item (optioneel hoofdstuk nog uit te werken in functie van beschikbare tijd, geschatte lengte: 2 blz.
\item In dit hoofdstuk zou ik alle literatuur willen bespreken die ik aan het begin van mijn thesis heb gelezen met het oog op het afbakenen van het onderwerp.  Dat zijn een hele hoop papers, maar er moet nog een goed stramien gevonden worden voor het structureren van de bespreking.
\end{itemize}
\item \textbf{Conclusion}(nog uit te werken, geschatte lengte: 2 blz.) 
\end{enumerate}

\section{Beschrijf in 10 lijnen de demo die je bij je verdediging zult geven.}
Ik heb een theoretische thesis en dus kan ik geen live demo geven. Ik ben van plan om mijn werkingsmethode uit te leggen aan de hand van enkele voorbeelden die ik in figuren zal uitwerken.
\section{Wat is er afgesproken met je begeleider en/of promotor om op te leveren als major milestone die het praktische werk van je thesis afsluit?}
Dit werd op de laatste vergadering besproken met de begeleiders. Ik stuurde na de vergadering een samenvattend verslag van de bespreking, maar de begeleiders antwoordden dat er op voorhand niet kan gezegd worden of de analyse van het FlipIt spel voldoende zal zijn omdat de indruk kan ontstaan dat dit resultaat eenvoudig te verkrijgen was. Er is dus niets afgesproken wat ik moet inleveren om mijn thesis af te sluiten.
%Als ik hier niet mee akkoord dan moet ik dit ook verduidelijken, en moet ik mijn standpunt verdedigen (in mijn tekst, en op de verdediging). 
Voorlopig werk ik verder af wat ik zelf al voor ogen had; namelijk het vinden van de nash equilibria en het schrijven van een related work over worm propagatie. \\
Ik stelde dezelfde vraag aan mijn promotor en kreeg het volgende antwoord: Mijn masterproef is in de eerste plaats mijn verantwoordelijkheid, ik ben verantwoordelijk voor wat ik indien en wat ik doe. De rol van mijn begeleiders en promotor is hierbij advies en richtlijnen geven.
Als ik het hiermee niet eens ben dan zorg ik voor de nodige argumenten en stel een sterke tekst op samen met een goede verdediging van mijn werk. 
Besluit: Er is dus vanuit promotor en werkleider geen concrete afspraak over wat ik moet inleveren om mijn thesis af te sluiten. Ik ben zelf van oordeel dat het uitwerken van de periodische strategie volstaat om tegemoet te komen aan de doelstelling van mijn masterproef en zal dit beargumenteren in de tekst van mijn masterthesis. 
\section{Evalueer de tijd die je in je thesis gestoken hebt in de voorbije 4 weken: veel te weinig / te weinig / voldoende / ruim voldoende / meer dan verwacht.}
De voorbije weken heb ik meer dan verwacht aan mijn thesis gewerkt. Momenteel heb  ik al 765 uren gewerkt. Hier zullen naar schatting nog eens 150 uren bijkomen. 
\section{Evalueer de vorderingen van je thesis t.o.v. het bereiken van de vooropgestelde doelstelling: ver achter op het schema / achter op het schema / goed op schema / vooruit op het schema / ver vooruit op het schema.}
Het geplande werk voor juni is uitgevoerd volgens schema. Tot 28 juni werk ik de tekst van hoofdstuk 5 verder af. In de week van 29 juni probeer ik mijn Nash equilibria hoofdstuk af te werken. 
\section{Stuur de ingevulde timesheet mee als bijlage.}




\end{document}
