\documentclass[a4paper, 11pt]{article}
\setlength{\topmargin}{-.5in}
\setlength{\textheight}{9in}
\setlength{\oddsidemargin}{.100in}
\setlength{\textwidth}{6.25in}
\usepackage [dutch] {babel}
\usepackage{graphicx}
\usepackage{amsmath}
\usepackage{fullpage}
\usepackage{listings}
\usepackage{color}
\usepackage{soul}
\usepackage{tikz}
\usepackage{hyperref}
\usepackage{parskip}

\definecolor{dkgreen}{rgb}{0,0.6,0}
\definecolor{gray}{rgb}{0.5,0.5,0.5}
\definecolor{mauve}{rgb}{0.58,0,0.82}

\lstset{frame=tb,
  language=Java,
  aboveskip=3mm,
  belowskip=3mm,
  showstringspaces=false,
  columns=flexible,
  basicstyle={\small\ttfamily},
  numbers=none,
  numberstyle=\tiny\color{gray},
  keywordstyle=\color{blue},
  commentstyle=\color{dkgreen},
  stringstyle=\color{mauve},
  breaklines=true,
  breakatwhitespace=true
  tabsize=3
}

\newcommand{\todo}[1]{{\huge \textcolor{green}{#1}}\\}
\newcommand{\comment}[1]{\textcolor{red}{#1}\\}
\newcommand{\viscomment}[1]{\textcolor{red}{#1}}

\begin{document}

\begin{titlepage}
\title{Vragenlijst thesis evaluatie}
\author{Sophie Marien}
\end{titlepage}

\maketitle

\section{Wat is de doelstelling van je thesis? (beschrijf deze doelstelling met eigen woorden in min 5, max 10 lijnen)}
Mijn thesis gaat over gametheorie en cybersecurity. Gametheorie is de studie van een spel waarbij er een interactie is tussen twee of meerdere spelers en de interactie die zal gebeuren afhangt van wat elke speler zal doen. Elke speler heeft een level van geluk voor de verschillende acties die gedaan kunnen worden. Met gametheory kunnen er security problemen gemodelleerd worden en kan er gekeken worden wat de best volgende zet is als defender van een systeem dat beschermd moet worden.


\section{Wat zijn de belangrijkste problemen die je moet oplossen? (max 5 lijnen)}
Een interessante link vinden tussen gametheory en cybersecurity die goed genoeg is om een thesis over te schrijven.
\section{Wat heb je tot op heden gedaan? (literatuur (wat precies?), leren werken met tools (welke?), analyse (van wat?), al iets ontworpen en/of geïmplementeerd (wat precies?), ...)}
Vooral literatuur doorgenomen en over gametheory gelezen en ingewerkt. Momenteel ben ik begonnen aan de coursera cursus over gametheory om dit topic volledig door en door te kennen. Via de infosessie heb ik ook leren werken met EndNote om een lijst bij te kunnen houden van al de literatuur die ik al heb doorgenomen. De literatuur die ik heb doorgenomen waren kleine papers over gametheory en cybersecurity en een grotere paper over FlipIT.
\section{Wat heb je gepland om te doen tijdens de volgende 4 weken?}
Volgende weken heb ik ingepland om mijn verloren uren van de eerste twee weken in te halen. Ik ga vooral nog verder lezen over gametheory en de lessen verder volgen van de coursera course. Ook moet ik mijn hypothese kunnen opstellen en duidelijk afbakenen. Daarna kan ik me dieper inwerken. 
\section{Tegen wanneer plan je om je thesis in te dienen (juni, september, januari)?}
Ik plan mijn thesis tegen Juni in te dienen.
\section{Stuur je ISP naar je begeleider en je promotor, en geef aan hoeveel examens je in de januari-zittijd en in de juni-zittijd zult hebben.}
In Januari zal ik drie examens hebben en in Juni zittijd maar een die ook twee weken na de thesis deadline zal vallen.
\section{Evalueer de tijd die je in je thesis gestoken hebt in de voorbije 4 weken: veel te weinig / te weinig / voldoende / ruim voldoende / meer dan verwacht.}
Voor de voorbije 4 (3?) weken heb ik er veel te weinig tijd ingestoken. Ik zit nog maar aan 25 uur. Ik ben twee weken in het buitenland geweest in de eerste twee weken van het academie jaar en ik was me ervan bewust dat ik al deze tijd terug zal moeten inhalen. De komende weken ga ik proberen om deze uren in te halen.
\section{Evalueer de vorderingen van je thesis t.o.v. het bereiken van de vooropgestelde doelstelling: ver achter op het schema / achter op het schema / goed op schema / vooruit op het schema / ver vooruit op het schema.}
Omdat ik wist dat ik de eerste weken niet aan mijn thesis zou kunnen werken zit ik op schema, maar dat betekent niet dat ik nu al goed bezig ben. 
\section{Stuur de ingevulde timesheet mee als bijlage.}
Ik kon de eerste vergadering niet komen en weet dus niet hoe ik de timesheet moet invullen. Ik heb een eigen timesheet aangemaakt om mijn uren bij te houden.
Voorlopig kom ik dus aan 24.15 uur voor deze week (weekend zit hier nog niet bij in).
\end{document}


