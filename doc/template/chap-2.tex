\chapter{The FlipIt game}
\label{cha:2}


In this chapter, we introduce the game \flip{FlipIt} \cite{FlipIt}. \flip{FlipIt} is a game introduced by van Dijk et al. First we explain the framework of FlipIt and it's important results. In the next section the formulas and assumptions are made that will be used throughout the paper. To understand how to model a FlipIt game with virus propagation it is important to get familiar with the concepts of the normal FlipIt game and it's notations.   

\section{The FlipIt game}
\flip{FlipIt} is a two-players game with a shared (single) resource that the players want to control as long as possible. The shared resource can be a password, a network or a secret key depending on the setting being modelled. In the rest of the paper we will call the two players the attacker, denoted by the subscript \textit{A} and the defender, denoted by subscript \textit{D}. 

The game begins at $t=0$ and continuous indefinitely ($t \rightarrow \infty $). The time in the game can be viewed as being continuous, but a discrete time can also be viewed. To get control over the resource, the players \textit{i} can flip the resource at any given time. A flip will be regarded as a move from a player \textit{i}. Each move will imply a certain cost $k_{i}$ and the cost can vary for each player. Both players will try to minimize their cost. By adding a cost, it will prevent players to move to frequently. \\

The unique feature of \flip{FlipIt} is that every move will happen in a stealthy way, meaning that the player has no clue that (his adversary) the other player has flipped the resource. For instance, the defender will not find out if the resource has already been compromised by the attacker, but he can only potentially know it after he flips the resource himself. The goal of the player is to maximize the time that he or she has control over the resource while minimizing total cost of the moves. A move can also result in a "wasted move", called a flop. It may happen that the resource was already under control by the defender. If the defender moves when he or she has already control over the resource, he or she would have wasted a move since it does not result in a change of ownership and a cost is involved. \\


\begin{figure}[hbtp]
\centering
\includegraphics[scale=0.8]{Images/DefFlipit}
\caption{A representation of a FlipIt game where both players are playing periodically and discrete. Every move or "flip" is indicated by a blue or orange circle. The blue and orange rectangles represent the amount of time one of the players is in control of the resource.}
\label{fig:FLipItDefault}
\end{figure}
\todo{verwijzen naar de figuur \ref{fig:FLipItDefault}}


The state of the resource is denoted as a time independent variable $C=C_{i}(t)$. 
$C_{D}(t)$ is either 1 if the game is under control by the defender and 0 if the game is under control by the attacker. For $C^{A}(t)$ it is visa versa, $C^{A}(t)= 1 - C^{D}(t)$.\\ \\
The game starts with defender being in control of the game, $C_{D}(0)= 1$. 

The players receive a benefit equal to the time of units that they were in possession of the resource minus the cost of making their moves. The cost of a player \textit{i} is denoted by $k_{i}$. 
The total gain of player \textit{i} is equal to the total amount of time that a player \textit{i} has owned the resource from the beginning of the game up to time \textit{t}. It is expressed as follows:
\begin{equation}\label{first}
G_{i}(t) = integraal [0][t] C_{i}(x) dx.
\end{equation}
The average gain of player \textit{i} is defined as:
\begin{equation}\label{first}
\gamma_{i}(t) = G_{i}(t)/t.
\end{equation}

Let $\beta_{i}(t)$ denote player's \textit{i} average benefit upto time \textit{t}:
\begin{equation}\label{first}
\beta_{i}(t) = \gamma_{i}(t) - k_{i}\alpha_{i}.
\end{equation}
This is equal to the fraction of time the resource has been owned by player \textit{i}, minus the cost of making the moves. ~$ \alpha_{i}$ defines the average move rate by player \textit{i} up to time \textit{t}.
\\

Because the players move in a stealthy way, there are different types of feedback that a player can get while moving:
\begin{itemize}
\item Non-adaptive (NA): The player does not receive any feedback during the game while flipping.
\item Last move (LM): When a player flips it will find out the exact time that the opponent played the last time.
\item Full History (FH): When a player flips it will find out the whole history of the opponents move.
\end{itemize}
The game can be extended by the amount of information that a player receives. It can also be possible for a player to get information at the start of the game. Both interesting cases are:
\begin{itemize}
\item Rate-of-play (RP): The player finds out the exact rate of play of the opponent.
\item Knowledge-of-strategy (KS): The player finds out the complete information of the strategy that the opponent is playing.
\end{itemize}

In our analyses of the FlipIt game with a virus propagation in section [], we assume the strategy of both players to be non-adaptive. None of the players has information of the strategy of the opponent. The defender will never know when the attacker will attack the network with a virus. Conversely, the attacker does not know how often the defender defends his network. 


\subsection{Strategies}
 
 In this subsection we elaborate about the strategy of FlipIt . For the other strategies of FlipIt we refer the reader to [the paper of FlipIt]
 \begin{table}
 \centering
 \begin{tabular}{ l | c  }
  \textbf{Categories} & \textbf{Classes of Strategies} \\
  \hline Non-adaptive (NA) & Exponential \\
  & Periodic \\
  & Renewal \\
  & General non-adaptive \\
  \hline Adaptive (AD) & Last move (LM) \\
  & Full History (FH) \\  
\end{tabular}
 \caption{Classes of strategies in FlipIt}
 \label{table:Strategies}
 \end{table}

There are two different kinds of strategies, the \textit{non-adaptive strategies} and the \textit{renewal strategies}. If there is no need for feedback for both of the players, we say that we have a non-adaptive strategy. Because the player does not receive any feedback during the game it will play in the same manner against every opponent. They are not dependent on the opponents movements. This means that they can already generate the time sequence for all the moves in advance.  But they can depend on some randomness because the non-adaptive strategies can be randomised. 
In this paper we will focus in the beginning on the non-adaptive strategies. Reasons behind this is that a player (defender or attacker) rarely knows what the strategies are of his opponent. [If the attacker wants to move stealthily, it might have limited attack options FLIPTHEM]. \todo{nog redenen zoeken}\\
A renewal strategy is a non-adaptive strategy where the time intervals between two consecutive moves are generated by a renewal process. \\

 \begin{description}
 \item Periodic
 \item Non-Arithmetic Renewal
 \item Exponential
 \end{description}
 



%%% Local Variables: 
%%% mode: latex
%%% TeX-master: "thesis"
%%% End: 
