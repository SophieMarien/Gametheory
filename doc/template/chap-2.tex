\chapter{The FlipIt game}
\label{cha:2}


\section{Extensions on FlipIt}

There a various possible ways to extend \flip{FlipIt}. For instance Laszka et al. extended the basic \flip{FlipIt} game to multiple resources. The incentive is that for compromising a system in a real case it needs more than just taking over one resource. An example is gaining access to a system and breaking the password. The model is called FlipThem \cite{FlipThem}. To ways of flipping the resources are used: the AND and the OR control model. In the AND model the attacker only controls the system if he controls all the resources of the system, whereas in the OR model the attacker only needs to compromise one resource to be in control of the entire system. The difference with FlipThem and this paper is that we introduce a Graph Model in the beginning.
Another extention on FlipIt is done by Pham [\todo{citatie needed voor Are We Compromised?}]. Beside the action Flip their is another action Test. The basic idea is to test with an extra action if the resource has been compromised or not. This action involves also an extra cost. This model is useful if somebody wants to know for example if his password has been compromised or wants to assess the periodic security of a system.  In [\todo{citatie van Mitigatie blabla en nog is mitigatie blabla}] \cite{MitigationNonTargeted} Laszka et al. 

kkk







In this section, we introduce the game \flip{FlipIt} \cite{FlipIt}. \flip{FlipIt} is a game introduced by .. .. and Rivest. First we explain the framework of FlipIt and after that the formulas and aannames that we will make for the game for during the whole paper.  

\section{The First Topic of this Chapter}
\flip{FlipIt} is a two-players game with a shared (single) resource that the players want to control as long as possible. The shared resource can be a password, a network or a secret key depending on the setting being modeled. In the rest of the paper we will call the players the Attacker and the Defender. To get the control over the resource, players can flip the resource at any given time. Each move will imply a certain cost. The unique feature of \flip{FlipIt} is that the move will happen in a stealthy way, meaning that the other player has no clue that the other player has flipped the resource. For instance, the defender will not find out if the resource has already been compromised by the attacker, but he can only potentially know it after he flips the resource himself. The goal of the player is to maximize the time that he or she has control over the resource while minimizing total cost of the moves. Players won't move to frequently. A move can also result in a "wasted move", in this paper called flop. It could be that the resource was for example already under control of the defender. If the defender moves when he or she has already the control over the resource, he or she would have a wasted move because it would not result in a change of ownership. 
 
Because the players move in a sthealty way, there are different types of feedback that a player can get while moving:
\begin{itemize}
\item Non-adaptive (NA): The player does not receive any feedback while flipping.
\item Last move (LM): The player finds out the exact time the opponent played the last time.
\item Full History (FH): The player finds out the complete history of the opponents move.
\end{itemize}
The game can be extended by the amount of information that a player receives. It can also be possible for a player to get information at the start of the game. Both interesting cases are:
\begin{itemize}
\item Rate-of-play (RP: The player finds out the exact rate of play of the opponent.
\item Knowledge-of-strategy (KS): The player finds out the complete information of the strategy that the opponent is playing.
\end{itemize}

In our assumption will the strategy of both players be non-adaptive. Non of the players has information of the strategy of the opponent. 


\subsection{An item}
A master thesis is never an isolated work. This means that your text must
contain references. On-line documents\cite{FlipThem} as well as
books\cite{DefendingAgainstUnknownEnemy} can be referenced. \cite{Craig2005}.

\section{Figures}
Figures are used to add illustrations to the text. The \fref{fig:logo} shows
the KU~Leuven logo as an illustration.
\begin{figure}
  \centering
  \includegraphics{logokul}
  \caption{The KU~Leuven logo.}
  \label{fig:logo}
\end{figure}

\section{Tables}
Tables are used to present data neatly arranged. A table is normally
not a spreadsheet! Compare \tref{tab:wrong} en \tref{tab:ok}: which table do
you prefer?

\begin{table}
  \centering
  \begin{tabular}{||l|lr||} \hline
    gnats     & gram      & \$13.65 \\ \cline{2-3}
              & each      & .01 \\ \hline
    gnu       & stuffed   & 92.50 \\ \cline{1-1} \cline{3-3}
    emu       &           & 33.33 \\ \hline
    armadillo & frozen    & 8.99 \\ \hline
  \end{tabular}
  \caption{A table with the wrong layout.}
  \label{tab:wrong}
\end{table}

\begin{table}
  \centering
  \begin{tabular}{@{}llr@{}} \toprule
    \multicolumn{2}{c}{Item} \\ \cmidrule(r){1-2}
    Animal    & Description & Price (\$)\\ \midrule
    Gnat      & per gram    & 13.65 \\
              & each        & 0.01 \\
    Gnu       & stuffed     & 92.50 \\
    Emu       & stuffed     & 33.33 \\
    Armadillo & frozen      & 8.99 \\ \bottomrule
  \end{tabular}
  \caption{A table with the correct layout.}
  \label{tab:ok}
\end{table}

\section{Lorem Ipsum}
This section is added to check headers and footers. So this chapter must at
least contain three pages. To make sure that we get the required amount,
the \textsf{lipsum} package isn't used but the text is put directly in the
text.

\subsection{Lorem ipsum dolor sit amet, consectetur adipiscing elit}
Sed nec tortor id felis tristique sodales. Nulla nec massa eu dui fermentum
tincidunt. Integer ullamcorper ante eget eros posuere faucibus. Nam id
ligula ut augue pulvinar vulputate id at purus. Aenean condimentum tortor
eu mi placerat eget eleifend massa mollis. Nam est mi, sagittis quis
euismod eget, sagittis in nibh. Proin elit turpis, aliquam et imperdiet
sed, volutpat eu turpis.

Pellentesque vel enim tellus, vitae egestas turpis. Praesent malesuada elit
non nisi sollicitudin non blandit lacus tincidunt. Morbi blandit urna at
lectus ornare laoreet. Suspendisse turpis diam, lobortis dictum luctus
quis, commodo at lorem. Integer lacinia convallis ultricies. Sed quis augue
neque, eu malesuada arcu. Nullam vehicula, purus vitae sagittis pulvinar,
erat eros semper massa, eu egestas nibh erat quis magna. Cras pellentesque,
nisl eu dapibus volutpat, urna augue ornare quam, quis egestas lectus nulla
a lectus.

Vivamus dictum libero in massa cursus sed vulputate eros imperdiet. Donec
lacinia, libero ac lobortis egestas, nibh dui ornare arcu, luctus porttitor
velit massa sit amet quam. Maecenas scelerisque laoreet diam, vitae congue
quam adipiscing vitae. Aliquam cursus nisl a leo convallis eleifend
fermentum massa porta. Nunc libero quam, dapibus dapibus molestie sit amet,
faucibus vel nunc.

\subsection{Praesent auctor venenatis posuere}
Sed tellus augue, molestie in pulvinar lacinia, dapibus non ipsum. Fusce
vitae mi vitae enim ullamcorper hendrerit eu malesuada est. Proin iaculis
ante sed nibh tincidunt vel interdum libero posuere. Vivamus accumsan metus
quis felis congue suscipit dapibus enim mattis. Fusce mattis tortor eget
ipsum interdum sagittis auctor id metus.

Integer diam lacus, pharetra sit amet tempor et, tristique non lorem.
Aenean auctor, nisi eu interdum fermentum, lectus massa adipiscing elit,
sed facilisis orci odio a lectus. Proin mi nibh, tempus quis porta a,
viverra quis enim. In sollicitudin egestas libero, quis viverra velit
molestie eget. Nulla rhoncus, dolor a mollis vestibulum, lacus elit semper
nisi, nec sollicitudin sem urna eu magna. Nunc sed est urna, euismod congue
mi.

\subsection{Cras vulputate ultricies venenatis}
Vivamus eros urna, sodales accumsan semper vel, lobortis sit amet mauris.
Etiam condimentum eleifend lorem, ullamcorper ornare lectus aliquet vitae.
Praesent massa enim, interdum sit amet semper et, venenatis ut elit.
Quisque faucibus, quam ac lacinia imperdiet, nulla neque elementum purus,
tempus rutrum justo massa porta sapien. Vestibulum ante ipsum primis in
faucibus orci luctus et ultrices posuere cubilia Curae; Sed ultrices
interdum mi, et rhoncus sapien rutrum sed.

Duis elit orci, molestie quis sollicitudin sed, convallis non ante.
Maecenas tincidunt condimentum justo, et ultricies leo tristique vitae.
Vestibulum quis quam non lectus dapibus eleifend a vitae nibh. Nam nibh
justo, pharetra quis iaculis consequat, elementum quis justo. Etiam mollis
lacinia lacus, nec sollicitudin urna lobortis ac. Nulla facilisi.

Proin placerat risus eleifend erat ultricies placerat. Etiam rutrum magna
nec turpis euismod consectetur. Phasellus tortor odio, lacinia imperdiet
condimentum sed, faucibus commodo erat. Phasellus sed felis id ante
placerat ultrices. Aenean tempor justo in tortor volutpat eu auctor dolor
mollis. Aenean sit amet risus urna. Morbi viverra vehicula cursus.

\subsection{Donec nibh ante, consectetur et posuere id, tempus nec arcu}
Curabitur a tellus aliquet ipsum pellentesque scelerisque. Etiam congue,
risus et volutpat rutrum, est purus dapibus leo, non cursus metus felis
eget ligula. Vivamus facilisis tristique turpis, ut pretium lectus luctus
eleifend. Fusce magna sapien, ullamcorper vitae fringilla id, euismod quis
ante.

Phasellus volutpat, nunc et pharetra semper, sem justo adipiscing mauris,
id blandit magna quam et orci. Vestibulum a erat purus, ut molestie ante.
Vestibulum ante ipsum primis in faucibus orci luctus et ultrices posuere
cubilia Curae; Proin turpis diam, consequat ut ullamcorper ut, consequat eu
orci. Sed metus risus, fringilla nec interdum vel, interdum eu nunc.
Suspendisse vel sapien orci.

\subsection{Morbi et mauris tempus purus ornare vehicula}
Mauris sit amet diam quam, eget luctus purus. Sed faucibus, risus semper
eleifend iaculis, mi turpis bibendum nisl, quis cursus nibh nisl sit amet
ipsum. Vestibulum tempor urna vitae mi auctor malesuada eget non ligula.
Nullam convallis, diam vel ultrices auctor, eros eros egestas elit, sed
accumsan arcu tortor eget leo. Vestibulum orci purus, porttitor in pharetra
eget, tincidunt eget nisl. Nullam sit amet nulla dui, facilisis vestibulum
dui.

Donec faucibus facilisis mauris ac cursus. Duis rhoncus quam sed nisi
laoreet eu scelerisque massa tincidunt. Vivamus sit amet libero nec arcu
imperdiet tempor quis non libero. Sed consequat dignissim justo. Phasellus
ullamcorper, velit quis posuere vulputate, felis erat tincidunt mauris, at
vestibulum justo lectus et turpis. Maecenas lacinia convallis euismod.
Quisque egestas fermentum sapien eu dictum. Sed nec lacus in purus dictum
consequat quis vel nisl. Fusce non urna sem. Curabitur eu diam vitae elit
accumsan blandit. Nullam fermentum nunc et leo dictum laoreet. Donec semper
varius velit vel fringilla. Vivamus eu orci nunc.

\section{Conclusion}
The final section of the chapter gives an overview of the important results
of this chapter. This implies that the introductory chapter and the
concluding chapter don't need a conclusion.

\lipsum[66]

%%% Local Variables: 
%%% mode: latex
%%% TeX-master: "thesis"
%%% End: 
