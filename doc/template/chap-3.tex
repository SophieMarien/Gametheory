%\chapter{Intoduction to GameTheory}
%\label{cha:1}
\documentclass[10pt]{article}
\begin{document}



\section{Literatuurstudie}
\begin{description}
\item Distributed Worm Simulation with a REalistic Internet 2005  \\
Modeling of congestions of network through worm propagation. Mathematical model focussing on the underlying network infrastructure.(diff no game theory) 
\item Of threats and Costs: A Game-theoretic approach to security risk managment 2013\\
Model network security of networks with a non-cooperative node through game theory. Attacker knows the defense strategies and the defender has knowledge of the possible attacks. Each actor considers the actions of the other before deciding to strive to optimize their own utility. (diff not stealthy)
\item Game theory meet network security and privacy (2013) \\
Chapter 3 addresses several games in game theory for modeling network security.
\item Game theoretic approach for cost-benefit analysis of malware proliferation prevention (..)
Introduces SIS and SIR together with 'patch', 'removal' and 'patch and removal'.
\end{description}

\subsection{What can be done in further research}
\begin{itemize}
\item Looking for the dynamics of the spread of the virus/worm limited by the bandwidth of the network links, BPG routing failure with high volume scan traffic
\end{itemize}
\section{Conclusion}

\section{Why Game Theory to model security problem}
Actors in a security protocol must follow the systems and some arbitrarily actors that are malisious and do not follow the protocol. [Bridging GameTheory and Cryptography]. Gametheoretic approach proposes a model where all the actors act with self-interest. 

\section{..}

Flip-it. Some authors have written other papers about flipit. One of them is the [Game theoretic approach for cost-benefit analysis of malware proliferation prevention]. 


Company networks are targeted. It costs a lot. They want to defend their company networks. No loss of data, integrity and confidentiality.  Many ways to attack a company. Virusses, trojans, worms, DOS, .. Hard to protect against every attacker.


%%% Local Variables: 
%%% mode: latex
%%% TeX-master: "thesis"
%%% End: 

\end{document}
