\chapter{Related work}
\label{cha:1}



\section{Related work}

Some related work, al dan niet game theorie

\begin{description}
\item Distributed Worm Simulation with a Realistic Internet 2005  \\
Modelling of congestions of network through worm propagation. Mathematical model focussing on the underlying network infrastructure.(diff no game theory) 
\item Of threats and Costs: A Game-theoretic approach to security risk management 2013\\
Model network security of networks with a non-cooperative node through game theory. Attacker knows the defence strategies and the defender has knowledge of the possible attacks. Each actor considers the actions of the other before deciding to strive to optimize their own utility. (diff not stealthy)
\item Game theory meet network security and privacy (2013) \\
Chapter 3 addresses several games in game theory for modelling network security.
\item Game theoretic approach for cost-benefit analysis of malware proliferation prevention (..)
Introduces SIS and SIR together with 'patch', 'removal' and 'patch and removal'.
\end{description}

\section{Why Game Theory to model security problem}
Actors in a security protocol must follow the systems and some arbitrarily actors that are malicious and do not follow the protocol. [Bridging Game Theory and Cryptography]. Game theoretic approach proposes a model where all the actors act with self-interest. 



%%% Local Variables: 
%%% mode: latex
%%% TeX-master: "thesis"
%%% End: 

