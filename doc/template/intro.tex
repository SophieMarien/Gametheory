\chapter{Introduction}
\label{cha:intro}
The first contains a general introduction to the work. The goals are
defined and the modus operandi is explained.
\todo{bib referenties in orde brengen}
\section{Introduction}

Security is an important asset in the computer science. Technology is growing fast and so are the malicious people. Defending a network of a company is not very easy. It makes use of firewalls, routers, IDS systems, virus scans, and so on. Protecting a company network is difficult job. They are often the victim of targetted attacks. In a security report of 2014, \todo{verwijzing naar report}, state that 80\% of the companies are the vitims of targeted attacks. 
Many corporate networks have to continuously defend themselves against outside invaders such as viruses and worms. The network administrator will try to keep the network to malware-free as possible. If still there is an intruder managed to penetrate the network then the network manager this intruder trying to get out as quickly as possible. This is not always easy. Especially when the intruders secretly sneak and then spread rapidly.
In this paper we will work further on the work made by Marten van Dijk, Ari Juels, Alina Oprea and Ronals L. Rivest \todo{verwijzing naar FlipIT} who wrote a report on the Game FlipIt. FlipIt is a the game of '' Stealthy Takeovers''. It models a game by means of two players, the attacker and the defender. Both can gain control over a single shared resource by flipping it. The most important property of the game is that the flipping happens stealthy. This means that the players have no clue about when the other player moves. The goal of the game is to maximise the time the player controls the resource minus the average cost of the flipping. 

\subsection{Motivation of the game}
%waarom FlipIt gebruiken en niet iets anders.
\subsection{Contributions and results}
\subsection{Conclusions}
The "I love you" virus is an example of a virus that spreads quickly. This virus propagates via mail systems. If someone opens an email with "I love you" virus in annex this virus spreads itself by sending a mail itself to everyone in your contact list. So the virus can multiply rapidly and eventually a business network shut down by the heavy traffic. In this example, there is a need human interaction to spread the virus to do. If no one opens the virus can not spread the mail.
Unfortunately, there are viruses that can spread without human interaction. These viruses are referred to as worms. A worm is also a computer program that replicates itself to spread to other computers so. Via a computer network, copies of the worm forwarded without an intermediary is used for. The worm will use vulnerabilities to infect other computers.
Most worms are designed to spread out and just try not to make any changes to the systems that they pass. These worms can still inflict damage by increased network traffic they generate. Worms that contain Harm damage a program to install a backdoor or a rootkit on the infected computers. Backdoors and rootkits ensure that future use can be made of the infected computers.
The Stuxnetworm is a very famous worm. Initially this worm spread via infected USB sticks and from then it could spread through the Internet to other computers. The purpose of the Stuxnetworm was broken to run the centrifuges in nuclear reactors. Many reactors have been infected. From the standpoint of the defender, it is very important to respond as quickly as possible so that the worm can not spread quickly.

%%% Local Variables: 
%%% mode: latex
%%% TeX-master: "thesis"
%%% End: 
