\chapter{Nash Equilibria}
\label{chapter:Nash}
%\documentclass[10pt]{article}
%\begin{document}

%%%%%%%%%%%%%%%%%%%%%%%%%%%%%%%%%%%%%%%%%%%%%%%%%%%%%%%%%%
%%%%%			Introduction Chapter 1				%%%%%%
%%%%%												%%%%%%
%%%%%												%%%%%%
%%%%%%%%%%%%%%%%%%%%%%%%%%%%%%%%%%%%%%%%%%%%%%%%%%%%%%%%%%

\section{Nash Equilibria}
-- rechtstreeks uit FlipIt paper --\\
As a second step, we are interested in finding Nash equilibria, points
for which neither player will increase his benefit by changing his rate of play. More
formally, a Nash equilibrium for the periodic game is a point $(\alpha_{0}^{*},\alpha_{1}^{*})$ such that
the defender's benefit $\beta_{0}(\alpha_{0},\alpha_{1}^{*}) $is maximized at $\alpha_{0}= \alpha_{0}^{*}$ and the attacker's benefit
$\beta_{1}(\alpha_{0}^{*},\alpha_{1}) $ is maximized at $\alpha_{1}= \alpha_{1}^{*}$ .
To begin with, some useful notation. We denote by opt0($\alpha_{1}$) the set of values (rates
of play $\alpha_{0}$) that optimize the benefit of the defender for a fixed rate of play $\alpha_{1}$ of the
attacker. Similarly, we denote by opt1($\alpha_{0}$) the set of values (rates of play $\alpha_{1}$) that optimize
the benefit of the attacker for a fixed rate of play $\alpha_{0}$ of the defender. The following
theorem specifies Nash equilibria for the periodic game and is proven in Appendix A. \\

Nash equilibria are points whith the property that neither player benefits by deviating in isolaition form equilibrium. We can compute Nash Equilibria for the periodic game as an intersection points of curvest opt0 and opt1. To determine opt0(a0) we need to compute the derivate of  $\beta_{0}(\alpha_{0},\alpha_{1}) $ for a fixed $\alpha_{1}$. We consider two cases:\\
-----

\textbf{Case: $\delta_{D} \leq \delta_{A} $\\

\begin{equation}
\beta_{D}(\alpha_{D},\alpha_{A}) = 1 - \dfrac{\delta_{D}}{2\delta_{A}} - \dfrac{k_{D}}{\delta_{D}} - \dfrac{d^{2}}{2\delta_{D}\delta_{A}} + \dfrac{d}{\delta_{A}}
\end{equation}
If we take the partial derivative we get the following result:
\begin{equation}\label{formdelta}
\frac{\partial \beta_{D}(\alpha_{D},\alpha_{A})}{\partial \alpha_{D}} = - \dfrac{1}{2\delta_{A}} + \dfrac{k_{D}}{\delta_{D}^{2}} + \dfrac{d^{2}}{2\delta_{D}^{2}\delta_{A}}
\end{equation}

Als we de formule \ref{formdelta} gelijkstellen aan 0 krijgen we:

\begin{equation}
\frac{\partial \beta_{D}(\alpha_{D},\alpha_{A})}{\partial \alpha_{D}} =0 ~~~~~~ =>~~~~~~ \delta_{D} = \sqrt{2\delta_{A}k_{D} + d^{2}}
\end{equation}

The function increases on $[0, \sqrt{2\delta_{A}k_{D} + d^{2}}]$ and is decreasing on $[\sqrt{2\delta_{A}k_{D} + d^{2}}, \infty]$ So we got a maximum on $\delta_{D} = minimum \{ \delta_{A}, \sqrt{2\delta_{A}k_{D} + d^{2}} \} $ \\
~~\\


\textbf{Case 2.A: $\delta_{D} \geq d+\delta_{A} \geq \delta_{A} $ }\\


 
\begin{equation}
 \beta_{D}(\alpha_{D},\alpha_{A})}{\partial \alpha_{D}} = \dfrac{\delta_{A}}{2\delta_{D}} + \dfrac{d}{\delta_{D}} - \dfrac{k_{D}}{\delta_{D}}
\end{equation}
\begin{equation}
\frac{\partial \beta_{D}(\alpha_{D},\alpha_{A})}{\partial \alpha_{D}} = -\dfrac{\delta_{A}}{2\delta_{D}^{2}} - \dfrac{d}{\delta_{D}^{2}} + \dfrac{k_{D}}{\delta_{D}^{2}}
\end{equation}

Als we de formule \ref{formdelta} gelijkstellen aan 0 krijgen we:
\begin{equation}
\frac{\partial \beta_{D}(\alpha_{D},\alpha_{A})}{\partial \alpha_{D}} =0 ~~~~~~ =>~~~~~~ \delta_{A} = 2(k_{D}-d) = dk_{D} - 2d
\end{equation}
\begin{description}
\item If $k_{D} < d$ 
\begin{description}
\item decreasing $ \delta_{A} < 2(k_{D} -d)$
\item increasing  $\delta_{A} > 2(k_{D} -d)$ \todo{equal}
\end{description}
\item If $k_{D} > d$ 
\begin{description}
\item increasing $ \delta_{A} < 2(k_{D} -d)$
\item decreasing  $\delta_{A} > 2(k_{D} -d)$ \todo{equal}
\end{description}
\end{description}
~~\\

\textbf{Case 2.B: $d+\delta_{A} \geq \delta_{D} \geq  \delta_{A} $} \\

\begin{equation}
\beta_{D}(\alpha_{D},\alpha_{A})}{\partial \alpha_{D}} = \dfrac{\delta_{A}}{2\delta_{D}} + \dfrac{d}{\delta_{D}} - \dfrac{k_{D}}{\delta_{D}} - \dfrac{(d-(\delta_{D} - \delta_{A}))^{2}}{d\delta_{D}\delta_{A}}
\end{equation}


\begin{equation}
\frac{\partial \beta_{D}(\alpha_{D},\alpha_{A})}{\partial \alpha_{D}} =  - \dfrac{1}{2\delta_{D}} + \dfrac{k_{D}}{\delta_{D}^{2}} + \dfrac{d^{2}}{2\delta_{D}^{2}\delta_{A}}
\end{equation}

Als we de formule \ref{formdelta} gelijkstellen aan 0 krijgen we:

\begin{equation}
\frac{\partial \beta_{D}(\alpha_{D},\alpha_{A})}{\partial \alpha_{D}} =0 ~~~~~~ =>~~~~~~ \delta_{D} = \sqrt{2\delta_{A}k_{D} + d^{2}}
\end{equation}


The function increases on $[0, \sqrt{2\delta_{A}k_{D} + d^{2}}]$ and is decreasing on $[\sqrt{2\delta_{A}k_{D} + d^{2}}, \infty]$ So we got a maximum on $\delta_{D} = minimum \{ \delta_{A}, \sqrt{2\delta_{A}k_{D} + d^{2}} \} $ \\


%%% Local Variables: 
%%% mode: latex
%%% TeX-master: "thesis"
%%% End: 

%\end{document}
