\chapter{Nash Equilibria}
\label{chapter:Nash}
%\documentclass[10pt]{article}
%\begin{document}

%%%%%%%%%%%%%%%%%%%%%%%%%%%%%%%%%%%%%%%%%%%%%%%%%%%%%%%%%%
%%%%%			Introduction Chapter 1				%%%%%%
%%%%%												%%%%%%
%%%%%												%%%%%%
%%%%%%%%%%%%%%%%%%%%%%%%%%%%%%%%%%%%%%%%%%%%%%%%%%%%%%%%%%

\section{Nash Equilibria}
%TODO-- rechtstreeks uit FlipIt paper --\\

In this chapter we are interested in finding the optimal strategies. To calculate the optimal strategies of both players we have to find the Nash Equilibria of the game. First the optimal functions are derived from the formulas in the previous chapter. From these piecewise functions we can derive the Nash Equilibria. \\

Nash equilibria are points with the property that neither player benefits by deviating in isolation from the equilibrium. We can compute Nash Equilibria for the periodic game as an intersection point of curves $opt_{D}$ and $opt_{A}$. 
\\
%As a second step, we are interested in finding Nash equilibria, points
%for which neither player will increase his benefit by changing his rate of play. 
More formally, a Nash equilibrium for the periodic game is a point $(\delta_{D}^{*},\delta_{A}^{*})$ such that
the defender's benefit $\beta_{D}(\alpha_{D},\delta_{A}^{*}) $is maximized at $\delta_{D}= \delta_{D}^{*}$ and the attacker's benefit
$\beta_{A}(\delta_{D}^{*},\delta_{A}) $ is maximized at $\delta_{A}= \delta_{A}^{*}$.
To begin with, some useful notation. We denote by $opt_{D}(\delta_{A}$) the set of values (rates
of play $\delta_{D}$) that optimize the benefit of the defender for a fixed rate of play $\delta_{A}$ of the
attacker. Similarly, we denote by $opt_{D}(\delta_{D}$) the set of values (rates of play $\delta_{A}$) that optimize
the benefit of the attacker for a fixed rate of play $\delta_{D}$ of the defender. \\

To determine $opt_{D}(\delta_{A})$ we need to compute the derivative of  $\beta_{D}(\delta_{D},\delta_{A}) $ for a fixed $\delta_{A}$.
 We consider two cases, where case 2 is divided into two subcategories. 
% Fig \todo{fig} shows all the cases.\\
%\begin{figure}[hbtp]
%\centering
%\includegraphics[scale=0.4]{Images/bestresp.png}
%\caption{This figure shows the all the cases with their subcategories. 'A' stands for Case 2.A: $\delta_{D} \geq d+\delta_{A} \geq \delta_{A}$ and 'B' stands for Case 2.B: $d+\delta_{A} \geq \delta_{D} \geq  \delta_{A} $ }
%\label{grafiekbestr}
%\end{figure}

\subsection{Determining the piecewise functions $opt_{D}(\delta_{A})$}


\subsection*{Case 1: $\delta_{D} \leq \delta_{A} $}
The benefit formula obtained in the previous chapter (formula \ref{Benfcase1:defender}) for the defender in this case is as follows:
\begin{equation*}
\beta_{D}(\delta_{D},\delta_{A}) = 1 - \dfrac{\delta_{D}}{2\delta_{A}} - \dfrac{d^{2}}{2\delta_{D}\delta_{A}} + \dfrac{d}{\delta_{A}}  - \dfrac{k_{D}}{\delta_{D}}
\end{equation*}

To know if the function decreases or increases we take the partial derivative of this formula for a fixed $\delta_{A}$:
\begin{equation*}\label{formdelta}
\frac{\partial \beta_{D}(\delta_{D},\delta_{A})}{\partial \delta_{D}} = - \dfrac{1}{2\delta_{A}} + \dfrac{k_{D}}{\delta_{D}^{2}} + \dfrac{d^{2}}{2\delta_{D}^{2}\delta_{A}}
\end{equation*}

The stationary points (maximum, minimum) can be found by setting the first derivative equal to zero and finding the roots of the resulting equation:
\begin{equation*}
\frac{\partial \beta_{D}(\delta_{D},\delta_{A})}{\partial \delta_{D}} =0 ~~~~~~ =>~~~~~~ \delta_{D} = \sqrt{2\delta_{A}k_{D} + d^{2}}
\end{equation*}

This leads to the following deduction given the sign of the coefficient of $\delta_{D}^{2}$: The function increases on $[0, \sqrt{2\delta_{A}k_{D} + d^{2}}]$ and is decreasing on $[\sqrt{2\delta_{A}k_{D} + d^{2}}, \infty]$. So we have a maximum at $\delta_{D} = min \{ \delta_{A}, \sqrt{2\delta_{A}k_{D} + d^{2}} \} $. Taking the minimum of the two values is needed because $\delta_{D}$ cannot be larger than $\delta_{A}$. \\
~~\\


\subsection*{Case 2.A: $\delta_{D} \geq d+\delta_{A} \geq \delta_{A} $ }

The benefit formula obtained in the previous chapter (formula \ref{benfcase2a:defender}) for the defender in this case is as follows:
\begin{equation*}
\beta_{D}(\delta_{D},\delta_{A})= \dfrac{\delta_{A}}{2\delta_{D}} + \dfrac{d}{\delta_{D}} - \dfrac{k_{D}}{\delta_{D}} = \dfrac{\delta_{A} + 2 (d-k_{D})}{\delta_{D}}
\end{equation*}

Given that $\delta_{D}$ is always positive, the benefit function can be either increasing or decreasing depending on the numerator of the above fraction. \\

For $\delta_{A} + 2(d-k_{D}) > 0$ the benefit will be always positive but decreasing, see figure \ref{ShapeUp}. 
The defender will always play if $\delta_{A} + 2(d-k_{D}) > 0$ for $k_{D} < d$ or $k_{D} > d$ because $\delta_{A}$ will be positive in either case. There is an edge case if $d=k_{D}$, which results in a benefit of $\beta_{D}(\delta_{D},\delta_{A})= \dfrac{\delta_{A}}{\delta_{D}}$. \\
\begin{figure}
\centering
\includegraphics[scale=0.5]{Images/ShapesUp.png} 
\caption{The benefit function is of the shape of 1/x if $\delta_{D}$ is always decreasing and if $\delta_{A} + 2(d-k_{D}) > 0$. }
\label{ShapeUp}
\end{figure}

For $\delta_{A} + 2(d-k_{D}) < 0$, the benefit will always be negative so the defender will not play. See figure \ref{ShapeDown}. This is independent of the value of $k_{D}$ and $ d$. \\

\begin{figure}
\centering
\includegraphics[scale=0.5]{Images/ShapeDown.png} 
\caption{The benefit function is of the shape of -1/x if $\delta_{D}$ is always increasing and if $\delta_{A} + 2(d-k_{D}) < 0$.}
\label{ShapeDown}
\end{figure}
%The derivative of the above formula for a fixed $\delta_{A}$ results in the following:
%\begin{equation*}
%\frac{\partial \beta_{D}(\delta_{D},\delta_{A})}{\partial \delta_{D}} = -\dfrac{\delta_{A}}{2\delta_{D}^{2}} - \dfrac{d}{\delta_{D}^{2}} + \dfrac{k_{D}}{\delta_{D}^{2}}
%\end{equation*}
%The obtain the stationary points the first derivative is set equal to zero and the roots of the resulting equation are found:
%\begin{equation*}
%\frac{\partial \beta_{D}(\delta_{D},\delta_{A})}{\partial \delta_{D}} =0 ~~~~~~ =>~~~~~~ \delta_{A} = 2(k_{D}-d) = dk_{D} - 2d
%\end{equation*}
%
%This leads to the following deduction:
%\begin{description}
%\item If $k_{D} \leq d$ 
%\begin{description}
% \item $\beta_{D}$ will be decreasing but always positive. If we minimize $\delta_{D}$ the value of $\beta_{D}$ will be higher. 
%\end{description}
%\item If $k_{D} > d$ 
%\begin{description}
%\item if $ \delta_{A} > 2(k_{D} -d)$, \\
%$\beta_{D}$ will be decreasing but always positive. If we minimize $\delta_{D}$ the value of $\beta_{D}$ will be higher. 
%\item if  $\delta_{A} = 2(k_{D} -d)$, \\
%the benefit of the defender will be $\beta_{D}=0$.
%\item if $\delta_{A} < 2(k_{D} -d)$, \\
%$\beta_{D}$ will be increasing but always negative. In this case the defender will not play. 
%\end{description}
%\end{description}
~~\\

\subsection*{Case 2.B: $d+\delta_{A} \geq \delta_{D} \geq  \delta_{A} $} 

The benefit formula obtained in the previous chapter  (formula \ref{benfcase2b:defender}) for the defender in this case is as follows:
\begin{equation*}
\dfrac{\beta_{D}(\alpha_{D},\alpha_{A})}{\partial \delta_{D}} = \dfrac{\delta_{A}}{2\delta_{D}} + \dfrac{d}{\delta_{D}} - \dfrac{k_{D}}{\delta_{D}} - \dfrac{(d-(\delta_{D} - \delta_{A}))^{2}}{2\delta_{D}\delta_{A}}
\end{equation*}

The derivative of the above formula for a fixed $\delta_{A}$ results in the following:
\begin{equation*}
\frac{\partial \beta_{D}(\alpha_{D},\alpha_{A})}{\partial \delta_{D}} =  - \dfrac{1}{2\delta_{A}} + \dfrac{k_{D}}{\delta_{D}^{2}} + \dfrac{d^{2}}{2\delta_{D}^{2}\delta_{A}}
\end{equation*}


The stationary points (maximum, minimum) can be found by setting the first derivative equal to zero and finding the roots of the resulting equation:

\begin{equation*}
\frac{\partial \beta_{D}(\delta_{D},\delta_{A})}{\partial \delta_{D}} =0 ~~~~~~ =>~~~~~~ \delta_{D} = \sqrt{2\delta_{A}k_{D} + d^{2}}
\end{equation*}


For case 2.B this leads to the following deduction, which results in the same formula as for case 1 but with a small difference for the value of $\delta_{D}$: The function increases on $[0, \sqrt{2\delta_{A}k_{D} + d^{2}}]$ and is decreasing on $[\sqrt{2\delta_{A}k_{D} + d^{2}}, \infty]$. Because $\delta_{D} \geq \delta_{A}$ there is a maximum on $\delta_{D} = maximum \{ \delta_{A}, \sqrt{2\delta_{A}k_{D} + d^{2}} \} $ or because $d+\delta_{A} \geq \delta_{D}$ there is also on $\delta_{D} = minimum \{ \delta_{A}+d, \sqrt{2\delta_{A}k_{D} + d^{2}} \} $. \todo{beter uitschrijven, deltaD moet tussen die twee waarden liggen}\\

\subsubsection{Best responses}
The optimum functions will be piecewise functions. There will be different optimum functions depending on $k_{D}$ and $d$. We distinguish for these two cases, three sub cases for different values of $\delta_{A}$. 

\subsection*{$k_{D} \leq d$}
Because $k_{D} \leq d$, the term $2(k_{D} - d)$ will always be negative. We point out that $\delta_{A}$ and $\delta_{D}$ are positive rates. 
\begin{itemize}
\item if $\delta_{A} < 2(k_{D} - d)$ \\
This means that $\delta_{A}$ has to be negative which is not possible. For case the defender will not play.
\item $\delta_{A} = 2(k_{D} - d)$ \\
$\delta_{A}$ is negative or equal to 0 so the attacker will not play. For case 1 and case 2.b the defender will also not play.
\item $\delta_{A} > 2(k_{D} - d)$ \\
Case 2.a it is increasing for every value $\delta_{A} \in [0,\infty]$.  For case 1 together with case 2.b the optimal benefit is achieved at rate $\delta_{D} = \sqrt{d^{2} + 2\delta_{A}k_{D}}$.
\end{itemize}


\subsubsection{$k_{D} > d$}
Because $k_{D} > d$, the term $2(k_{D} - d)$ will always be positive. We point out that $\delta_{A}$ and $\delta_{D}$ are positive rates.
\begin{itemize}
\item if $\delta_{A} < 2(k_{D} - d)$ \\
From case 2.a it follows that the benefit of the defender increases. From case 1 and case 2.b together the optimal benefit of the defender is achieved at rate $\delta_{D} = \sqrt{d^{2} + 2\delta_{A}k_{D}}$.
\item $\delta_{A} = 2(k_{D} - d)$ \\
From case 2.a it follows that $\beta_{D}(\delta_{D},\delta_{A})=0$, for all $\delta_{A} \in [0,2(k_{D} - d)]$. From case 1 and case 2.b together the optimal benefit for the defender is achieved for all rates $\delta_{D} \in [0, \sqrt{d^{2} + 2\delta_{A}k_{D}}]$.
\item $\delta_{A} > 2(k_{D} - d)$ \\
For case 2.a the benefit is decreasing. From case 1 and case 2.b the best strategy for the defender is not playing at all. 
\end{itemize}
\todo{dat laatste nog eens nakijken}

From this analyses we can compute $opt_{D}(\delta_{A})$ for two different cases as: \\

For case  $k_{D} \leq d$:
 \begin{displaymath}
  opt_{D}(\delta_{A}) = \left\{
     \begin{array}{lr}
       0, & \delta_{A} < 2(k_{D} - d)\\
       0, & \delta_{A} = 2(k_{D} - d) \\
       \sqrt{d^{2} + 2\delta_{A}k_{D}}, & \delta_{A} > 2(k_{D} - d)
     \end{array}
   \right.
\end{displaymath}

For case $k_{D} > d$ :
 \begin{displaymath}
  opt_{D}(\delta_{A}) = \left\{
     \begin{array}{lr}
       \sqrt{d^{2} + 2\delta_{A}k_{D}}, & \delta_{A} < 2(k_{D} - d)\\
       \big[0,\sqrt{d^{2} + 2\delta_{A}k_{D}}\big], & \delta_{A} = 2(k_{D} - d) \\
       0, & \delta_{A} > 2(k_{D} - d)
     \end{array}
   \right.
\end{displaymath}


%*****************************************************************
%
% optimum functies voor beta A
%
%*********************************************************************
\subsection{Determining the piecewise functions $opt_{A}(\delta_{D})$}
To start with we only consider the case where $d < \delta_{D}$, because if $d > \delta_{D}$ the benefit of the defender \todo{nakijken, def of att} is always 1. \\
To determine the Nash equilibria we also need to determine $opt_{A}(\delta_{D})$ by computing the derivative of $\beta_{A}(\delta_{D},\delta_{A})$ for a fixed $\delta_{D}$. We consider again two cases, where case 2 is divided into two subcategories: \\

\subsection*{Case 1: $\delta_{A} \geq \delta_{D}$}

The benefit formula obtained in the previous chapter for this case is as follows:
\begin{equation*}
\beta_{A}(\delta_{D},\delta_{A}) =\dfrac{\delta_{D}}{2\delta_{A}} - \dfrac{k_{A}}{\delta_{A}} + \dfrac{d^{2}}{2\delta_{D}\delta_{A}^{2}} - \dfrac{d}{\delta_{A}}
\end{equation*}
The derivative for a fixed $\delta_{D}$ is as follows:
\begin{equation*}
\dfrac{\partial \beta_{A}(\delta_{D},\delta_{A})}{\partial \delta_{A}} = -\dfrac{\delta_{D}}{2\delta_{A}^{2}} + \dfrac{k_{A}}{\delta_{A}^{2}} - \dfrac{d^{2}}{2\delta_{D}\delta_{A}^{2}} + \dfrac{d}{\delta_{A}^{2}}
\end{equation*}
The stationary points (maximum, minimum) can be found by setting the first derivative equal to zero and finding the roots of the resulting equation:
\begin{equation*}
\frac{\partial \beta_{A}(\delta_{D},\delta_{A})}{\partial \delta_{D}} =0 ~~~~~~ =>~~~~~~ 2k_{A} = \dfrac{(\delta_{D}-d)^{2}}{\delta_{D}}
\end{equation*}

It follows that $\beta_{A}(\delta_{D},\cdot)$ is increasing if $2k_{A} < (\delta_{D} - d)^{2} / \delta_{D}$ and decreasing if $2k_{A} > (\delta_{D} - d)^{2} / \delta_{D}$. \\

\subsection*{Case 2.A: $\delta_{D} \geq d+\delta_{A} \geq \delta_{A} $ }
The benefit formula obtained previously for this case is as follows:
\begin{equation*}
\beta_{A}(\delta_{D},\delta_{A}) =1- \dfrac{\delta_{A}}{2\delta_{D}} - \dfrac{k_{A}}{\delta_{A}} - \dfrac{d}{\delta_{D}}
\end{equation*}
The derivative for a fixed $\delta_{D}$ is as follows:
\begin{equation*}
\dfrac{\partial \beta_{A}(\delta_{D},\delta_{A})}{\partial \delta_{A}} = \dfrac{-1}{2\delta_{D}} + \dfrac{k_{A}}{\delta_{A}^{2}}
\end{equation*}
The stationary points (maximum, minimum) can be found by setting the first derivative equal to zero and finding the roots of the resulting equation:
\begin{equation*}
\frac{\partial \beta_{A}(\delta_{D},\delta_{A})}{\partial \delta_{D}} =0 ~~~~~~ =>~~~~~~ \delta_{A} = \sqrt{2\delta_{D}k_{A}}
\end{equation*}
It follows that $\beta_{A}(\delta_{D},\cdot)$ is increasing on $[0,\sqrt{2k_{A}\delta_{D}}]$ and decreasing on $[\sqrt{2k_{A}\delta_{D}}, \infty]$ and thus has a maximum on $\delta_{A} = maximum \{\delta_{D}, \sqrt{2k_{A}\delta_{D}} \} $. The maximum between $\delta_{D}$ and $ \sqrt{2k_{A}\delta_{D}}$ is needed because $\delta_{A} $ cannot exceed $\delta_{D}$ in this case. \\

\subsection*{Case 2.B: $d+\delta_{A} \geq \delta_{D} \geq  \delta_{A} $} 

The benefit formula obtained previously for this case is as follows: 
\begin{equation*}
\beta_{A}(\delta_{D},\delta_{A}) = 1 - \dfrac{\delta_{A}}{2\delta_{D}} - \dfrac{d}{\delta_{A}} - \dfrac{k_{A}}{\delta_{A}} + \dfrac{(d-(\delta_{D}-\delta_{A})^{2}}{2\delta_{D}\delta_{A}} 
\end{equation*}
The derivative for a fixed $\delta_{D}$ is as follows:
\begin{equation*}
\dfrac{\partial \beta_{A}(\delta_{D},\delta_{A})}{\partial \delta_{A}} = -\dfrac{\delta_{D}}{2\delta_{A}^{2}} + \dfrac{k_{A}}{\delta_{A}^{2}} - \dfrac{d^{2}}{2\delta_{D}\delta_{A}^{2}} + \dfrac{d}{\delta_{A}^{2}}
\end{equation*}
The stationary points (maximum, minimum) can be found by setting the first derivative equal to zero and finding the roots of the resulting equation:
\begin{equation*}
\frac{\partial \beta_{A}(\alpha_{D},\alpha_{A})}{\partial \delta_{D}} =0 ~~~~~~ =>~~~~~~ 2k_{A} = \dfrac{(\delta_{D}-d)^{2}}{\delta_{D}}
\end{equation*}

it follows that $\beta_{A}(\delta_{D},\cdot)$ is increasing if $2k_{A} < (\delta_{D} - d)^{2} / \delta_{D}$ and decreasing if $2k_{A} > (\delta_{D} - d)^{2} / \delta_{D}$. This is the same result as in case 1.\\


\subsubsection{Best responses}
The optimum functions will be piecewise functions. We distinguish three cases for different values of $\delta_{D}$ and $k_{A}$. 


For this term $\dfrac{(\delta_{D}-d)^{2}}{\delta_{D}} $ , $d$ has to be bigger than  $\delta_{D}$ because the cost $k_{A}$ cannot be negative. This was an assumption that was already made, because the benefit of the defender will always be 1 if $d$ is bigger than  $\delta_{D}$.
\begin{itemize}
\item if $2k_{A} < \dfrac{(\delta_{D}-d)^{2}}{\delta_{D}} $ \\
Then for case 1 and case 2.b the benefit of the defender is increasing. From case 2.a follows that the optimal benefit for the attacker is achieved at the rate $\delta_{A} = \delta_{D}$
\item if $2k_{A} = \dfrac{(\delta_{D}-d)^{2}}{\delta_{D}} $ \\
From case 1 and case 2.b it follows that $\beta_{D}(\delta_{D},\delta_{A})=0$, for all $\delta_{A} \in [0,\dfrac{(\delta_{D}-d)^{2}}{2k_{A}})]$. From case 2.a the optimal benefit for the defender is achieved for all rates $\delta_{A} \in [0, \delta_{D}]$.
\item if $2k_{A} > \dfrac{(\delta_{D}-d)^{2}}{\delta_{D}} $ \\
All decreasing.
\end{itemize}

This brings us to the following opt function for : \todo{juiste resultaten hier nog invullen}
From this analyses we can compute $opt_{D}(\delta_{A})$ for two different cases as:
 \begin{displaymath}
  opt_{A}(\delta_{D}) = \left\{
     \begin{array}{lr}
       \delta_{D}, & \delta_{A} < 2(k_{D} - d)\\
       \left[ 0, \delta_{D} \right],  & \delta_{A} = 2(k_{D} - d) \\
       0 & \delta_{A} > 2(k_{D} - d)
     \end{array}
   \right.
\end{displaymath}
\\

The Nash Equilibria can be found as the intersection points of the piecewise functions $opt_{A}(\delta_{D})$ and $opt_{D}(\delta_{A})$. For this we have to compare the function in terms of the relationship of the players move cost. We distinguish three cases: $k_{A} < k_{D} , ~k_{A} > k_{D} $ and $k_{A} = k_{D}$. 
\todo{ kA met kD vergelijken en nash evenwichten vinden}
\todo{ een voorbeeld invullen met specifieke waarden}