\chapter{Formula}
\label{cha:9}
%\documentclass[10pt]{article}
%\begin{document}

%%%%%%%%%%%%%%%%%%%%%%%%%%%%%%%%%%%%%%%%%%%%%%%%%%%%%%%%%%
%%%%%			Introduction Chapter 1				%%%%%%
%%%%%												%%%%%%
%%%%%												%%%%%%
%%%%%%%%%%%%%%%%%%%%%%%%%%%%%%%%%%%%%%%%%%%%%%%%%%%%%%%%%%


%------------------------------------------------%
%            Intro Game Theory 					 %
%------------------------------------------------%
\section{something}
\label{Cha:9:inrto}

Periodic Game with delay for the attacker:

\textbf{Case 1:} $\delta_{D} \leq \delta_{A} $(The defender plays at least as fast as the attacker.) \\

Let $r = \dfrac{\delta_{D}}{ \delta_{A} }$. The intervals between two consecutive defender's moves have length $\delta_{D}$. Consider a given defender move interval. The probability over the attacker's phase selection that the attacker moves in this interval is r. Given that the attacker moves within the interval, he moves exactly once within the interval (since $\delta_{D} \leq \delta_{A} $) and his move is distributed uniformly at random. \\

The expected period of attacker control within the interval would be r/2, without considering the delay. \\

However, because of the delay, the maximal time of control is reduced to $\delta_{D}-d$. If we consider a duration of $\delta_{D} \cdot \delta_{A}$ the attacker will play $\delta_{D}$ times. If the attacker plays soon enough it will get a gain of $\dfrac{\delta_{D}-d}{2}$ in $\delta_{D}-d$ of the cases. In \textit{d} cases it will receive a gain of zero. This is the case were the duration of the delay causes the defender to play before the attacker can get control over the resource. 
So the gain of the attacker can be expressed as follows:
\begin{equation}\label{first}
Gain =  \dfrac{\delta_{D}-d}{2} \cdot (\delta_{D}-d) + 0 \cdot d = \dfrac{\delta_{D}-d}{2} \cdot (\delta_{D}-d)
\end{equation}

The benefit of the attacker can be expressed as follows

\begin{equation}\label{first}
\beta_{A}(\alpha_{D},\alpha_{A}) = \dfrac { (\delta_{D}-d) ^{2}} {2 \cdot \delta_{D}  \delta_{A}} + k_{A} \cdot \alpha_{A}
\end{equation}
\begin{equation}\label{first}
\beta_{A}(\alpha_{D},\alpha_{A}) = \dfrac { \delta_{D}} {2 \cdot \delta_{A}} + k_{A} \cdot \alpha_{A} + \dfrac{d}{\delta_{A}} + \dfrac{d^{2}}{2 \cdot \delta_{A} \delta_{D}}
\end{equation}
\\

The benefit of the defender is then:

\begin{equation}\label{first}
\beta_{D}(\alpha_{D},\alpha_{A}) = 1 - \dfrac { (\delta_{D}-d) ^{2}} {2 \cdot \delta_{D} \cdot \delta_{A}} + k_{D} \cdot \alpha_{D}
\end{equation}
\begin{equation}\label{first}
\beta_{D}(\alpha_{D},\alpha_{A}) = 1 - \dfrac { \delta_{D}} {2 \cdot \delta_{A}} + k_{D} \cdot \alpha_{D} + \dfrac{d}{\delta_{A}} + \dfrac{d^{2}}{2 \cdot \delta_{A} \delta_{D}}
\end{equation}


\textbf{Case 2:} $\delta_{A} \leq \delta_{D} $ (The attacker plays at least as fast as the defender.) \\

Let $r = \dfrac{\delta_{D}}{ \delta_{A} }$. The intervals between two consecutive attacker's moves have length $\delta_{A}$. Consider a given attackers move interval. The probability over the attacker's phase selection that the defender moves in this interval is $\dfrac{\delta_{D}}{ \delta_{A} } = (1/r)$. Given that the defender moves within the interval, he moves exactly once within the interval (since$\delta_{A} \leq \delta_{D} $) and his move is distributed uniformly at random. \\

If we consider a duration of $\delta_{A} \cdot \delta_{D} $ there is a probability of $\dfrac{\delta_{A} } {\delta_{D}} $ that the defender moves within the interval of the attacker. The defender will then receive an average gain of $\dfrac{\delta_{A}} {2} $. There is $1- \dfrac{\delta_{A} } {\delta_{D}} $ probability that the defender will not move in the interval of the attacker and so the defender will receive no gain. The benefit can be expressed as follows when the  defender plays $\delta_{D}$ times during a duration of $\delta_{A} \cdot \delta_{D}$:
\begin{equation}\label{first}
\beta_{D}(\alpha_{D},\alpha_{A}) = \dfrac { 1} {\delta_{A} \delta_{D}} \cdot \delta_{D} \cdot [ \dfrac{\delta_{A}}{\delta_{D}} \cdot \dfrac{\delta_{A}}{2}+[ 1-\dfrac{\delta_{A}}{ \delta_{D}}] \cdot 0 ]
\end{equation}
\begin{equation}\label{first}
\beta_{D}(\alpha_{D},\alpha_{A}) = \dfrac {\delta_{A} }{2 \cdot \delta_{D}} $$same as the FlipIt solution$$
\end{equation}

However, because of the delay, the maximal time of control of the defender is increased by d. In other words, the defender has some benefit time of d before the attacker really gains control over the resource, meaning that the attacker gains control only after $\delta_{A}+d$ instead of after $\delta_{A}$. So, when the defender plays, with a probability of $\dfrac{\delta_{A} } {\delta_{D}} $, the expected gain of the defender's control in this interval would be more than half of the period $\delta_{A} $: it is $\dfrac{\delta_{A} + d}{2}$. There is $1- \dfrac{\delta_{A} } {\delta_{D}} $ probability that the defender will not move in the interval of the attacker but because of the delay the defender will receive a gain of d. So the benefit of the defender can be expressed as:
\begin{equation}\label{first}
\beta_{D}(\alpha_{D},\alpha_{A}) = \dfrac { 1} {\delta_{A} \delta_{D}} \cdot \delta_{D} \cdot [ \dfrac{\delta_{A} }{\delta_{D}} \cdot \dfrac{\delta_{A} + d}{2}+[ 1-\dfrac{\delta_{A}}{ \delta_{D}}] \cdot d ]
\end{equation}
\begin{equation}\label{first}
\beta_{D}(\alpha_{D},\alpha_{A}) = \dfrac {\delta_{A} - d}{2 \cdot \delta_{D}} + \dfrac{d}{\delta_{A}} 
\end{equation}

%nce the defender has played, it takes 1 time until the attacker plays and the defender loses control again. Therefere, the expected period of attacker control within the interval would be 1/2.1/0, without considering the delay.

%%% Local Variables: 
%%% mode: latex
%%% TeX-master: "thesis"
%%% End: 

%\end{document}
