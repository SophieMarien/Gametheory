\chapter{Intoduction to GameTheory}
\label{cha:1}
%\documentclass[10pt]{article}
%\begin{document}

%%%%%%%%%%%%%%%%%%%%%%%%%%%%%%%%%%%%%%%%%%%%%%%%%%%%%%%%%%
%%%%%			Introduction Chapter 1				%%%%%%
%%%%%												%%%%%%
%%%%%												%%%%%%
%%%%%%%%%%%%%%%%%%%%%%%%%%%%%%%%%%%%%%%%%%%%%%%%%%%%%%%%%%


In the following paragraph an introduction to game theory is given based on the work of 
{leyton2008essentials} 
and 
{Coursera}. 
For a more detailed and full introduction to game theory, the reader is referred to 
{leyton2008essentials}. 
%------------------------------------------------%
%            Intro Game Theory 					 %
%------------------------------------------------%
\section{Intro Game Theory}
\label{Cha:1:Intro.Game.Theory}



%begin over dat gametheorie handig is in de economie

Game theory studies the interaction between independent and self-interested agents. It is a mathematical way of modelling the interactions between two or more agents where the outcomes depend on what everybody does and how it should be structured to lead to good outcomes. For this reason it is very important for economics and also for politics, biology, computer science, philosophy and a variety of other disciplines.  \\
%Every agent has different levels of happiness for the different outcomes.
%self interested meaning

One of the assumptions underlying game theory is that the players of the game, the agents, are independent and self-interested. This does not necessarily mean that they want to harm other agents or that they only care about themselves. 
%utility function meaning 
Instead it means that each agent has preferences about the states of the world he likes. These preferences are mapped to natural numbers and are called the utility function. The numbers are interpreted as a mathematical measure to tell you how much an agent likes or dislikes the states of the world. \\
It also explains the impact of uncertainty. When an agent is uncertain about a distribution of outcomes, his utility will describe the expected value of the utility function with respect to the probability of the distribution of the outcomes. For example: with 0.7 probability it will be 7 degrees outside and 0.3 probability it will be 10 degrees. The agent can have a different opinion about that distribution versus another distribution. (\todo{uitleggen aan de hand van een voorbeeld}).\\
%Cooperative and non cooperative games
In a decision game theoretic approach an agent will try to act in such a way to maximise his expected or average utility function. It becomes more complicated when two or more agents want to maximise their utility and whose actions can affect each other utilities. This kind of games are referred to as non cooperative game theory, where the basic modelling unit is the group of agents. The individualistic approach, where the basic modelling is only one agent, is referred as cooperative game theory. 

There are two standard representations for games. The first one is the Normal Form. The second one is the Extensive Form.

In the following list a couple of terms that will be used throughout the paper.
\begin{description}
\item \textit{Players}: players are referred as the ones who are the decision makers. It can be a person, a company or an animal.
\item \textit{Actions}: actions are what the player can do. 
\item \textit{Outcomes}:  
\item \textit{Utility function}: the utility function is the mapping of the level of happiness of an agent about the state of the world to natural numbers.
\item \textit{Strategies}: A strategy is the combination of different actions. A pure strategy is only one action.
\end{description}

A game in game theory consists of multiple agents and every agent has a set of actions that he can play. 



%Nash equilibrium



% --------------- example of a game -----------------%


%------------------------------------------------%
%            Intro about virusses				 %
%------------------------------------------------%
\section{Virusses}

Stealth
Regin's developers put considerable effort into making it highly inconspicuous. Its low key nature means it can potentially be used in espionage campaigns lasting several years. Even when its presence is detected, it is very difficult to ascertain what it is doing. Symantec was only able to analyze the payloads after it decrypted sample files.

It has several ''stealth'' features. These include anti-forensics capabilities, a custom-built encrypted virtual file system (EVFS), and alternative encryption in the form of a variant of RC5, which isn't commonly used. Regin uses multiple sophisticated means to covertly communicate with the attacker including via ICMP/ping, embedding commands in HTTP cookies, and custom TCP and UDP protocols
%http://www.symantec.com/connect/blogs/regin-top-tier-espionage-tool-enables-stealthy-surveillance
Ways of defending a network:
\begin{itemize}
\item Self-defending networks: The next generation of network security
\item Honeynet games: a game theoretic approach to defending network monitors

\end{itemize}
Many network security threats today are spread over the Internet. The most common include:

Viruses, worms, and Trojan horses
Spyware and adware
Zero-day attacks, also called zero-hour attacks
Hacker attacks
Denial of service attacks
Data interception and theft
Identity theft

%http://www.ists.dartmouth.edu/library/258.pdf Email Virus Propagation Modeling and Analysis
%Cliff C. Zou∗, Don Towsley†, Weibo Gong∗
%∗Department of Electrical & Computer Engineering
%†Department of Computer Science
%Univ. Massachusetts, Amherst
%Technical Report: TR-CSE-03-04

Computer virus through mail. 
Though virus spreading through email is an old technique, it is still effective and is widely used by
current viruses and worms. Sending viruses through email has some advantages that are attractive to
virus writers:
 Sending viruses through email does not require any security holes in computer operating systems
or software.
 Almost everyone who uses computers uses email service.
 A large number of users have little knowledge of email viruses and trust most email they receive,
especially email from their friends [28][29].
 Email are private properties like post office letters. Thus correspondent laws or policies are required
to permit checking email content for detecting viruses before end users receive email [18].

Send a email with malicious attachment. Only again infected if attachment again opened. Thus this is the action of attacking every neighbour node + also can attack again the node where the virus was coming from.
There are also email viruses were the malicious program is hidden in the txt and the attachment does not need to be opened. 

%http://www.cisco.com/web/offer/gist_ty2_asset/Cisco_2014_ASR.pdf p49
%http://repo.hackerzvoice.net/depot_madchat/vxdevl/papers/avers/2004-35.pdf
%http://www.mcafee.com/us/resources/white-papers/foundstone/wp-managing-malware-outbreak.pdf



\subsection{Malware}
%Does a company network faces lot of malware? what is the cost ?
Relevant researches:
\begin{itemize}
%http://ants.iis.sinica.edu.tw/3BkMJ9lTeWXTSrrvNoKNFDxRm3zFwRR/17/04483668.pdf
\item How Viruses and worm can be detected. Difference between UDP en TCP worm propagation
\end{itemize}




%%%%%%%%%%%%%%%%%%%%%%%%%%%%%%%%%%%%%%%%%%%%%%%%%%%%%%%%%%
%%%%%			Conclusion Chapter 1				%%%%%%
%%%%%												%%%%%%
%%%%%												%%%%%%
%%%%%%%%%%%%%%%%%%%%%%%%%%%%%%%%%%%%%%%%%%%%%%%%%%%%%%%%%%
\section{Conclusion}
\label{Cha:1:Conclusion}
The final section of the chapter gives an overview of the important results
of this chapter. This implies that the introductory chapter and the
concluding chapter don't need a conclusion.


%%% Local Variables: 
%%% mode: latex
%%% TeX-master: "thesis"
%%% End: 

%\end{document}
