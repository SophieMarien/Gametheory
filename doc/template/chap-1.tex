\chapter{Intoduction to GameTheory}
\label{cha:1}
%\documentclass[10pt]{article}
%\begin{document}


\section{Intro Game Theory}

%begin over dat gametheorie handig is in de economie
In the following paragraph an introduction to game theory is given based on the work of \todo{werk van Essentials of Game Theory} and \todo{Coursera game theory}.
Game theory studies the interaction between independent and self-interested agents. For this reason it is very important for economics and also for politics, biology, computer science, philosophy and a variety of other disciplines. It is a mathematical way of modelling the interactions between two or more agents where the outcome depends on what everybody does and how it should be structured to lead to good outcomes. \\
%Every agent has different levels of happiness for the different outcomes.
A self-interested independent agent does not necessarily mean that they want to harm other agents or that they only care about themselves. Each agent has preferences about the states of the world he likes. These preferences are mapped to natural numbers and are called the utility function. The numbers are interpreted as a mathematical measure to tell you how much an agent likes or dislikes the states of the world. It also explains the impact of uncertainty. When an agent is uncertain about a distribution of outcomes, his utility will describe the expected value of the utility function with respect to the probability of the distribution of the outcomes. For example: with 0.7 probability it will be 7 degrees outside and 0.3 probability it will be 10 degrees. The agent can have a different opinion about that distribution versus another distribution. (\todo{uitleggen aan de hand van een voorbeeld}).\\
 

A game in game theory consists of multiple agents and every agent has a set of actions that he can play. 
%self interested meaing

%utility function meaning

%Cooperative and non cooperative games

% example of a game


\section{Other}
In this chapter an introduction to gametheory will be given with the formulas that will be used troughout this paper. We start with the basics of gametheory. People that have a background in gametheory can skip this chapter.


\section{Virusses}

Many network security threats today are spread over the Internet. The most common include:

Viruses, worms, and Trojan horses
Spyware and adware
Zero-day attacks, also called zero-hour attacks
Hacker attacks
Denial of service attacks
Data interception and theft
Identity theft

%http://www.ists.dartmouth.edu/library/258.pdf Email Virus Propagation Modeling and Analysis
%Cliff C. Zou∗, Don Towsley†, Weibo Gong∗
%∗Department of Electrical & Computer Engineering
%†Department of Computer Science
%Univ. Massachusetts, Amherst
%Technical Report: TR-CSE-03-04

Computer virus through mail. 
Though virus spreading through email is an old technique, it is still effective and is widely used by
current viruses and worms. Sending viruses through email has some advantages that are attractive to
virus writers:
 Sending viruses through email does not require any security holes in computer operating systems
or software.
 Almost everyone who uses computers uses email service.
 A large number of users have little knowledge of email viruses and trust most email they receive,
especially email from their friends [28][29].
 Email are private properties like post office letters. Thus correspondent laws or policies are required
to permit checking email content for detecting viruses before end users receive email [18].

Send a email with malicious attachment. Only again infected if attachment again opened. Thus this is the action of attacking every neighbour node + also can attack again the node where the virus was coming from.
There are also email viruses were the malicious program is hidden in the txt and the attachment does not need to be opened. 

%http://www.cisco.com/web/offer/gist_ty2_asset/Cisco_2014_ASR.pdf p49
%http://repo.hackerzvoice.net/depot_madchat/vxdevl/papers/avers/2004-35.pdf
%http://www.mcafee.com/us/resources/white-papers/foundstone/wp-managing-malware-outbreak.pdf


\subsection{What are my topics}
\begin{itemize}
\item Security, Costs, Cybersecurity
\item Viruses, kinds
\item Gametheory
\item Flip-it
\item Flip-it multiple resources
\item 
\end{itemize}

\subsection{Malware}
%Does a company network faces lot of malware? what is the cost ?
Relevant researches:
\begin{itemize}
%http://ants.iis.sinica.edu.tw/3BkMJ9lTeWXTSrrvNoKNFDxRm3zFwRR/17/04483668.pdf
\item How Viruses and worm can be detected. Difference between UDP en TCP worm propagation
\end{itemize}






\section{Conclusion}
The final section of the chapter gives an overview of the important results
of this chapter. This implies that the introductory chapter and the
concluding chapter don't need a conclusion.


%%% Local Variables: 
%%% mode: latex
%%% TeX-master: "thesis"
%%% End: 

%\end{document}
