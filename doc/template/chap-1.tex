%\chapter{Intoduction to GameTheory}
%\label{cha:1}
\documentclass[10pt]{article}
\begin{document}

In this chapter an introduction to gametheory will be given with the formulas that will be used troughout this paper. We start with the basics of gametheory. People that have a background in gametheory can skip this chapter.


\section{Virusses}

Many network security threats today are spread over the Internet. The most common include:

Viruses, worms, and Trojan horses
Spyware and adware
Zero-day attacks, also called zero-hour attacks
Hacker attacks
Denial of service attacks
Data interception and theft
Identity theft

%http://www.ists.dartmouth.edu/library/258.pdf Email Virus Propagation Modeling and Analysis
%Cliff C. Zou∗, Don Towsley†, Weibo Gong∗
%∗Department of Electrical & Computer Engineering
%†Department of Computer Science
%Univ. Massachusetts, Amherst
%Technical Report: TR-CSE-03-04

Computer virus through mail. 
Though virus spreading through email is an old technique, it is still effective and is widely used by
current viruses and worms. Sending viruses through email has some advantages that are attractive to
virus writers:
 Sending viruses through email does not require any security holes in computer operating systems
or software.
 Almost everyone who uses computers uses email service.
 A large number of users have little knowledge of email viruses and trust most email they receive,
especially email from their friends [28][29].
 Email are private properties like post office letters. Thus correspondent laws or policies are required
to permit checking email content for detecting viruses before end users receive email [18].

Send a email with malicious attachment. Only again infected if attachment again opened. Thus this is the action of attacking every neighbour node + also can attack again the node where the virus was coming from.
There are also email viruses were the malicious program is hidden in the txt and the attachment does not need to be opened. 

%http://www.cisco.com/web/offer/gist_ty2_asset/Cisco_2014_ASR.pdf p49
%http://repo.hackerzvoice.net/depot_madchat/vxdevl/papers/avers/2004-35.pdf
%http://www.mcafee.com/us/resources/white-papers/foundstone/wp-managing-malware-outbreak.pdf


\subsection{What are my topics}
\begin{itemize}
\item Security, Costs, Cybersecurity
\item Viruses, kinds
\item Gametheory
\item Flip-it
\item Flip-it multiple resources
\item 
\end{itemize}

\subsection{Malware}
%Does a company network faces lot of malware? what is the cost ?
Relevant researches:
\begin{itemize}
%http://ants.iis.sinica.edu.tw/3BkMJ9lTeWXTSrrvNoKNFDxRm3zFwRR/17/04483668.pdf
\item How Viruses and worm can be detected. Difference between UDP en TCP worm propagation. Difference for the propagation speed. 
\end{itemize}



Purpose thesis: model a worm propagation with adaptations of Flip-It. Flip-It cannot address the evaluation of individual nodes. Flip-It with multiple resources has not addressed the fact that a virus does not need to compromise the whole network. A subpart of the network can already cause problems for the company. Data leackages. 
So the virus can propagate 


\section{Conclusion}
The final section of the chapter gives an overview of the important results
of this chapter. This implies that the introductory chapter and the
concluding chapter don't need a conclusion.


%%% Local Variables: 
%%% mode: latex
%%% TeX-master: "thesis"
%%% End: 

\end{document}
