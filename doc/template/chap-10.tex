\chapter{Introduction}
\label{cha:10}
%\documentclass[10pt]{article}
%\begin{document}

%%%%%%%%%%%%%%%%%%%%%%%%%%%%%%%%%%%%%%%%%%%%%%%%%%%%%%%%%%
%%%%%			Introduction Chapter 10			%%%%%%
%%%%%												%%%%%%
%%%%%												%%%%%%
%%%%%%%%%%%%%%%%%%%%%%%%%%%%%%%%%%%%%%%%%%%%%%%%%%%%%%%%%%

\section{Introduction}

In this era where digitalization becomes prominent in every aspect of our lives, where technology is growing fast and where businesses are always under attack, security becomes an issue of increasing complexity. Security is needed to protect websites, servers, applications, data, operating systems and other assets that need protection in a computer network. Without security, there is no protection to keep somebody out of a system. It is the same as leaving the door of your house wide open for everyone to come in. \\
%Businesses can have confidential information on clients. Through data leakage, confidential information can be lost and possibly used by the competitor. Businesses wants to meet their service-level agreements. They will protect themselves against disruption that can be caused by DOSS attacks. Ultimately, system and network security helps protecting a business's reputation, which is one of its most important assets. 

Why is it so important to keep a system secure?  Many businesses store confidential information, which can be lost through data leakage and can possibly be abused by competitors. Also, disruption caused by distributed denial of service (DDoS) attacks, may result in businesses failing to meet their service-level agreements. Ultimately, computer and network security helps protecting a business against various kind of threats. \\
%\todo{waarom?  :  lekken van informatie: je hebt info over klanten en moet hun privacy beschermen, je eigen data is geld waard voor concurrenten, DOSS attacks: je will je service-level agreements kunnen nakomen, ..}
% A hacker will be a person that seeks exploits or weaknesses in a system or network in order to gain access.  Many of those attacks have a different cause. Some of the attacks by a hacker can be benign, others can be harmful. There are various ways to break into a system. Viruses, worms, spyware and other malware are the number two of the top external threats that a business faces [security report kaspersky 2014]. (number one is Spam). Furthermore these kind of threats also causes the greatest percentage in loss of data. These threats will infect the network by means of a virus that will propagate through the network. Most of the attacks are Advanced Persistent Threats (APT).\\

A particular kind of threat is an Advanced Persistent Threat (APT). An APT is a multi-faceted, continuous and targeted cyber attack that is designed to penetrate a network or a system in a stealthy way and can stay undetected for a long period of time. It is different and more severe than a conventional threat. A conventional threat will not attack any particular target. An APT is persistent and will keep on trying to attack its victim. It operates silently and stealthily, to prevent detection. This makes it so hard to protect a network or a system against an APT. 

%Bruce Schneier describes an APT as something different and stronger than a conventional threat: ''\textit{A conventional hacker or criminal is not interested in any particular target. He wants a thousand credit card numbers for fraud, or to break into an account and turn it into a zombie, or whatever. Security against this sort of attacker is relative; as long as you're more secure than almost everyone else, the attackers will go after other people, not you. An APT is different; it's an attacker who - for whatever reason - wants to attack you. Against this sort of attacker, the absolute level of your security is what's important. It does not matter how secure you are compared to your peers; all that matters is whether you're secure enough to keep him out}'' - Bruce Schneier \cite{APTBruce}.\\


%\subsubsection{Complication}
%Since it is so difficult to protect a system or a network against APT's, researchers have been looking for effective ways to predict in advance which defence strategy might be the better one. 
There are a number of key strategies an organisation can apply to defend itself against APT's: awareness, whitelisting, system administration, network segregation, dynamic content checking and patch management. Nevertheless the combination of all these elements benefits from being complemented by other defence strategies to protect oneself against stealthy takeovers. One possible way to study the impact of stealthy takeovers and to determine practical recommendations for defenders is through game theory.\\
Game theory is gaining increasing interest as an effective technique to model and study cyber security problems. It is common to model cyber security problems as a game with two players, an attacker and a defender. There are, however, games that have more players e.g. when a third party is involved \cite{fengstealthy}. This paper focusses on a game with two players. The actions available to the attacker and the defender correspond respectively to the attacks on the system and to take defensive measures that protect the system. \\

Many security games that bridge game theory and cyber security have already been investigated, so finding a new game can be challenging. This paper builds on a relatively new paper where the assumption of stealthiness is fairly unique, giving some interesting results.\\ 
 The paper is from researchers at RSA, van Dijk et al,  who presented a game-theoretic framework to model computer security scenarios called ``FlipIt'' \cite{FlipIt}. They study the specific scenario where a system or network is repeatedly taken over completely by an attacker. This take-over is not immediately detected by the defender. It is a two-player game where the attacker and the defender are competing to get control over a shared resource. Neither player knows who is currently in control of the resource until they move. In FlipIt every move involves a cost and gives them immediate control over the resource. The attacker will try to maximise the time that he controls the network, while the defender will try to maximise the time that the network is free of malware. \\
 But what if the attacker moves and it takes some time before the attacker gets full control over the resource? FlipIt does not take into account that a move may not be instantaneous, but has a certain delay. Consider for example a network with different nodes (laptops, datacenters) as a resource. The attacker drops a virus on one of the nodes and waits until this virus infects the whole network. The attacker will only be in control of the resource when a sufficiently large amount of nodes of the network are infected. In this paper we present an adaptation of FlipIt to model a game where the moves of the attacker are not instantaneous. The formalization for this game starts from the model of non-adaptive continuous basic FlipIt game where players use a periodic strategy with a random phase.   \\

\subsubsection{Research questions}

%This paper proposes the adaptation of the FlipIt formulas as presented in \cite{FlipIt} such as to take the delay for virus propagation into account. In the next section we first present the original FlipIt game. Then section \ref{ch:flipitvirus} presents the FlipIt game with virus propagation. Section \ref{ch:extendedWork} presents some related work. Section \ref{ch:conclusion} concludes the paper and presents avenues for further research.

This paper adapts the model presented in \cite{FlipIt} so as to take the delay for virus propagation into account.
This leads us to the following research questions:
\begin{itemize}
\item How can we incorporate the notion of delay in the game-theoretical analysis of the Flip-It game for a periodic strategy?
\item Does the resulting model allow an optimal defence strategy against an attacker? 
\end{itemize}

\subsubsection{Contributions and results}
The following contributions are made in this paper:
\begin{itemize}
\item[-] We propose an addition to the basic FlipIt model to model a scenario where the moves by the attacker will not be instantaneous. We extend the FlipIt game to a game wherein the attacker flips with a delay. The attacker only compromises the system if sufficient nodes in the network are infected. 
\item[-] The periodic case of FlipIt is modelled with a delay, resulting in optimum functions and Nash equilibria. \todo{punt herschrijven en meer nadruk leggen op het vinden van de Nash equilibria}
\item[-] Based on our results we can give practical recommendations to take measures against advanced attacks.  The modelling of FlipIt with a propagation delay will give a guideline for both the attacker and the defender on how to respectively plan attacks or network clean-ups.
\item[-] This work includes an overview of different techniques used to calculate the propagation delay of malware (depending on the network layout). \todo{en het wordt ook toegepast op de formules}
\item[-] It presents a method to calculate the speed of the propagation of a worm in a network independent of the topology of the network.
\end{itemize}


While it may seem trivial to extend the basic FlipIt model with a propagation delay, it's mathematical treatment is not. In the paper of Laznka et all. \citep{FlipThem} it seems that even a small extension adds to the already significant mathematical complexity. While the FlipIt game is quite symmetric, by adding a propagation delay the mathematical complexity rises.
% First a formula to calculate the benefit of a basic FlipIt was derived to introduce the propagation delay. To calculate the benefit for each case depending on the periods of the attacker, defender and the delay, the intervals where not independent any more as in the basic FlipIt game. The previous intervals had to be taken into account to calculate the benefit of the players in its own interval. 

%\subsubsection{Conjunctures and open problems}
%
%\begin{description}
%\item[-] Analyse other strategies like renewal strategies instead of only periodic strategies.
%\item[-] The defender can chose to Flip one node of the network or a subset instead of all the nodes.
%\item[-] Give the defender a delay. It takes a while before a patch is made after the discovery of a vulnerability.
%\end{description}

\subsubsection{Overview of the thesis}

The organisation of this paper is the following.  An introduction to cyber security and game theory is given in chapter \ref{Chapter1:Intro.Game.Theory} to allow the reader to become familiar with the kind of threats that are in the scope of this work and the game theoretic concepts that will be further used in the paper. In the same chapter the FlipIt framework is summarized with its most important conclusions. The chapter concludes with related work on FlipIt and further clarifies the contributions of this paper compared to existing work. 
Chapter \ref{chapter2:FlipIt with virus propagation} first introduces the adaptations made to the original FlipIt model to model a FlipIt game with virus propagation. Subsequently formulas are derived to model a FlipIt game with a virus propagation for the specific case where players play a periodic strategy with a random phase. \\
In Chapter \ref{chapter:Nash} the formulas are further analysed to determine if there is a nash equilibrium. 
In order to provide a clear perspective on delays, chapter \ref{chapter4: Worm propagation} gives an overview of the various methods of propagation and worm propagation models. It presents a method to calculate the speed of the propagation of a worm in a network independent of the topology of the network.
Finally chapter \ref{chapter5:conclusion} discusses the main results and provides directions for further research.

%In this paper we want to focus on situations where a computer network is attacked by an APT. These threats will infect the network by means of a virus that will propagate through the network. 
%ð  Dat is geen vraag.
%
% 
%Kijk naar de opdracht van je thesis. De vraag zou bv. kunnen zijn:
%General research question: Does the application of game theory help in defining a good defense (or attacker) strategy ?
%This question is to large, so we reduce the question to …
%Specific research question:
%-         Is it possible to incorporate the notion of delay in the game-theoretical analysis of the Flip-It game ?
%
%-        Can we calculate a Nash Equilibrium for a FlipIt Game with delay ?
%
%-        Does this allow us to determine an optimal defense strategy against an attacker ?
%
% 
%Je matrix-berekening beantwoordt de vraag:
%-        How can we calcuate the expected duration for a node’s infection/the entire network infection ?
%
%-        Can we calculate this node per node ?
%
% 
%Complication
%Wat je schrijft is veel te algemeen. Maak het heel concreet en toegespitst op jouw specifieke bijdrage ?
%ð  De complicatie is dat de FlipIt game geen rekening houdt met de delay


%%%% Local Variables: 
%%%% mode: latex
%%%% TeX-master: "thesis"
%%%% End: 

