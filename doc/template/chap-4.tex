\chapter{Models for the Delay}
\label{chapter4: Worm propagation}



The formalisation of the FlipIt game with a delay relies on some value  \textit{d} that represents the time needed to infect a sufficient number of nodes in a network after the initial infection. This chapter provides the reader with more insight on how to calculate the value of this parameter \textit{d}.  \\

The  spreading of malware has already been extensively researched. Because of the many different types of propagation methods it is hard to define a single model that can model all of them. Modelling the spread of malware depends on two key factors: the method used for the propagation and the graph of the network in which the malware will spread itself. \\
Viruses and worms are the types of malware that are the most researched. Since the spreading of viruses requires human interaction, their propagation delay depends on (hard to predict) human behaviour. Worms on the other hand, spread without human intervention, and their propagation is therefore easier to model. Since the purpose of this chapter is only to illustrate how a delay can be calculated, we will limit this chapter to propagation models of worms. \\

The overview of this chapter will be as follows: Section \ref{methodsofpropagation} presents an overview of the most frequently used propagation methods by worms. These propagation methods will be covered by different kinds of models in section \ref{modelsforpropagation} presents models to determine the time a worm needs to infect a sufficient number of nodes. These results can be used for the completion of parameter \textit{d}. \todo{wat wordt er met de vorige zin bedoeld?} Finally, section \ref{eigenmatrixmethode}, introduces an easy method to calculate the delay of the propagation of a worm. % Different kind of graphs to model a computer network are introduced. 

%This chapter is based on the following papers: \cite{importantjournal}, \cite{OnWorms2005survey}, \cite{GameTheorApprCostBenefitAnalyses} and \cite{SecurelistAPT}.


\section{Methods of propagation}
\label{methodsofpropagation}

In the context of malware propagation, there are two kinds of APT's. First, there are APT's that launch an attack on just one target node of a network. The mechanism these APT's use to propagate malware is the dropper mechanism. The dropper is the initial attack vector that compromises the single targeted node of the system. The second kind of APT's target multiple nodes. These APT's also use a dropper mechanism but have an additional mechanism for self-propagation in order to propagate themselves to the multiple targeted nodes of the system. If the APT uses virus spreading, the virus infects one node on the network and has to wait for human interaction to spread. As such the spreading speed depends on the human interaction, and modelling virus propagation therefore requires modelling human behaviour. If an APT uses a worm propagation method it will spread by itself after it has been dropped on the network. Given the additional complexity of incorporating the human factor in a spreading method and that this chapter is merely meant as illustration on calculating the delay for use in the game theoretic approach, this chapter will be limited to worm propagation models for APT's. \\

Infection by worms start by dropping the worm on the network. Different dropping mechanisms can be used, whereby the initial attack can be random or targeted at a specific node. Frequently used mechanisms are USB sticks (given to a specific person or left behind to be picked up by a random person), email, or malicious software through fishing or trojan horses. To determine the total delay, however, the propagation strategy is of higher importance than the drop mechanism. Propagation is achieved by determining the next nodes to spread the worm to. The following are common propagation methods: 
\begin{description}
%\item Scanning methods: A worm tries to guess (scan) the address of potential target nodes to infect. It can use distinct ways of scanning.
%% random, localized, topological or hit-list scanning.
%\begin{itemize}

\item \textbf{Selective Random scanning:} The worm randomly  selects a part of the selected IP address space instead of scanning the whole address space. The reserved address blocks and the unassigned addresses are excluded from the address space. The rate of success for randomly chosen IP addresses is very low, but it is easy to implement. Example of such worms are \textit{Code Red} \citep{OwnInternetSI} and \textit{Slammer} \citep{moore2003inside}. \todo{Voorbeeeld als mensen het idee niet snappen?}

\item \textbf{Localized scanning:} A worm that uses localized scanning will scan for hosts in the local address space. This method is used by the \textit{Code Red II} \citep{OwnInternetSI} and \textit{Nimda worm} \citep{OwnInternetSI}.
%\item \textbf{Topological scanning:} Address information stored in the victim machines are used to locate new targets. This method used by the `Morris' worm.

%\item \textbf{Hit-list scanning:} With the method of hit-list scanning, a short-list of vulnerable systems is made beforehand.  This is done to speed up the spread of worms at the initial stage. --This list consists of potentially vulnerable machines that are gathered beforehand and targeted first when the worm is released. An extreme case for the hit-list-scanning worms is a flash worm, which gathers all vulnerable machines into the list.--

\item \textbf{Sequential scanning:}
With sequential scanning, the worm will scan the IP addresses sequentially. This means that once a vulnerable host is compromised, it will look for IP addresses that are near to this host. For example, the address of the host is A, the next addresses that the worm will scan are A+1, or A-1. This method is used by the \textit{Blaster worm} \citep{zou2006performance}.

\item \textbf{DNS random scanning:} Another strategy is a kind of strategy in which the DNS infrastructure is used to locate a new target address. The IP address table from a DNS server is acquired from DNS records. 
The speed of a DNS scanning worm in the IPv6 internet is comparable to the speed of an IPv4 random scanning worm. 
%There are some difficulties with this scanning technique. 
%First it is difficult to obtain the whole address space from the DNS records. Secondly, the worm has to carry the database, which can slow down the propagation if the database is to big. Last, t
The IP addresses stored in the address table are only hosts with public domain names. This propagation method is used by \textit{MyDoom} \citep{kamra2005effect}.

\item \textbf{Routable scanning:} The worms using routable scanning acquire target IP addresses based on the routing information in a network. Through  the BGP routing tables they can scan the routable address space. This method is three times faster than a traditional worm that uses random scanning. Examples are \textit{Spyb0t} or \textit{network Bluepill}. 
%--a ''routing worm'' [20]. Zou et al. designed two types of routing worms [20]. One type, based on Class-A (x.0.0.0/8) address allocations, is thus called 'ClassA routing worms.'' Such worms can reduce the scanning space to 45.3\% of the entire IPv4 address space. The other type, based on BGP routing tables, is thus called''BGP routing worms'' Such worms can reduce the scanning space to only about 28.6\% of the entire IPv4 address space.--
%\end{itemize}   

\item \textbf{Topology-based Worms} Email and other client application worms:An email worms uses the email systems to find email addresses to propagate. Other client applications can include: Internet Relay Chat (IRC), Instant Messenger (IM), and a variety of peer-to-peer
file sharing systems, which have been used by worms to propagate in a similar way as email worms. For example, the \textit{Kak worm} \citep{Kakworm} is a JavaScript computer worm that spread itself by exploiting a bug in Outlook Express. % Another example is `GhostNet'

%Pikachu worm: The virus was mainly spread through Microsoft Outlook email attachments. The email containing the attached virus propagated through infected users by sending itself to all contacts in the user's Outlook address book. [5]
%\item \textbf{self stopping worm:} reduces speed to avoid detection. Atak worm or self stopping worm \cite{VirusChangePropSpeed}.
%\item \textbf{network sharing} `Shamoon'
%\item \textbf{shared files or spreading through SMB} (shared message block or  Common Internet File System (CIFS)) Flame
%\item \textbf{Zero-day vulnerabilities} \begin{description}
%\item Crouching Yeti is hardly a sophisticated campaign. For example, the attackers used no zero-day exploits, only exploits that are widely available on the Internet. But that did not prevent the campaign from staying under the radar for several years. \url{http://www.kaspersky.com/internet-security-center/threats/crouching-yeti-energetic-bear-malware-threat}
\end{description}

 

\section{Models for worm propagation}
\label{modelsforpropagation}
To use the models for worm propagation in the model of the FlipIt game with propagation delay it is sufficient to know the propagation speed of the worm and determine after how many time units a sufficient number of nodes in the network is infected. This will be equal to the \textit{d} parameter in our FlipIt model. \\


%Scan based or Topology based
A node in a network can be in three different states: susceptible, infected or removed. A susceptible node is a node that is vulnerable to infection. An infected node is a node that has been compromised and can infect other nodes. A removed node is dead or immune, which means it cannot be infected again by worms. With these three states three main propagation models are proposed: SI, SIR and SIS. In the SI model, a node that has once been infected, stays infected. In the SIR model, an infected node can be removed afterwards. This node cannot become infected again. In the SIS model a node can become susceptible again after it has been infected. Based on these three models various other models have been proposed. \todo{referenties naar andere modellen} We are only interested in models that will allow us to find the time required for a worm to infect a sufficient amount of nodes.
Table \ref{tree} gives an overview of the most common propagation methods with the given type. This table comes from the most recent survey \citep{wang2014modeling}, which encompasses results of older surveys. Moreover, the survey explicitly focuses on propagation methods, while other papers also focus on other aspects such as  detection mechanisms and containment systems. All models that are based on SI model type (Susceptible-Infected) are good, because in our model every node that is infected will stay infected. Only the Local Preference and Email worms models will not be explained because they involve human interaction. OSN model is done by means of simulation. The SIR AAWP model is also interesting because the parameter that involves the R part of the model can be deleted.  \\ \todo{laatste zinnen zijn wazig: Het deel erboven over SIS enzo was duidelijk maar hier worden andere modellen er bij de haren bijgesleurd. Als we die niet bekijken, gewoon kort zeggen dat er andere modellen zijn (e.g. SIR AAWP, Email worms, ....) maar dat deze niet worden bekeken ofzo?}



%The methods of propagation can be divided into different model types. SI, SIS and SIS... uitbreiden.. We are only interested in models that are based on SI model type (Susceptible-Infected), because in our model every node that is infected will stay infected. 


%The propagation models of diseases are applicable to the propagation of worms. SI, SIS and SIR. The SI model or also SEM (Simple Epidemic Model) is a model where the hosts in the network only have two states: Susceptible or Infected. Once they are infected they stay that way. The second model, SIS, is a model where the hosts can return to the Susceptible state, which in turn can become infected again. SIR is a model where the hosts can go in a Removed state. Once it is in a Removed state the hosts can not be infected any more. \\


%In this chapter we are only interested in the SI model. To apply the propagation models on FlipIt we need to know the propagation speed of the virus, without hosts becoming susceptible again. All hosts must stay infected. For this we only look at SI models. In table \ref{tree} the taxonomy of the propagation methods are given and the type of model. The General Epidemic model and Two-factor model are based on the Simple Epidemic Model with the difference that they both have extra parameters to model the Removed state.  We discuss a couple of SI models and then introduce a method of out own.\\

For more information about other models the reader is referred to the following papers with other most common propagation models: \cite{wang2014modeling}, \cite{OnWorms2005survey}, \cite{xiang2009propagation} and \cite{serazzi2004computer}.

%In this section we only list the models that are the most interesting from the perspective of calculating the delay, namely the SI type models.
%SI, SIR, SIS assume $\beta$ to be constant.

%Tree model figure \ref{tree}. We introduce homogeneous models. Network topology is mixed. 

\begin{figure}
\centering
%\includegraphics[scale=0.4]{Images/tabel3.png} 
%\caption{Taxonomy of worm modelling. Image based on taxonomy given in \cite{wang2014modeling}}.
%\label{tree}
%\end{figure}
\includegraphics[scale=0.55]{Images/tableworms.png}
\caption{Taxonomy of worm modelling. Image based on taxonomy given in \cite{wang2014modeling}}.
\label{tree}
\end{figure}


\subsection{Simple Epidemic Model}
%Codered paper gebruikt voor model
A Simple epidemic model is another name for the general SI model. This model assumes that each node in the network can be either susceptible or infected. 
Once a node is infected it will stay infected. Every node in the network has the same chance to be infected. 
The simple epidemic model is considered to be of a fixed size, meaning that no nodes are added to the network or removed. The model for a fixed population is as follows:

\begin{equation}
\dfrac{d I(t)}{dt} = \beta I(t)[N-I(t)]
\end{equation} 
where \textit{I(t)} is the number of infected nodes at time \textit{t}, $\beta$ is the propagation rate, and \textit{N} is the number of nodes in the network. In the beginning, $t=0$, \textit{I(0)} nodes are infected. All the other nodes, $N - I(0)$, in the network are susceptible. 
The solution of this equation is the following logistic curve:
\begin{equation}
I = \dfrac{e^{\beta(t-T)}}{1+e^{\beta(t-T)}}
\label{SIdelay}
\end{equation}
where \textit{T} is a time parameter representing the point of maximum increase in the growth.\\


%\begin{figure}[hbtp]
%\centering
%\includegraphics[scale=0.5]{Images/SEMmodel.png}
%\caption{Simple Epidemic Model. The x-coordinate is the propagation time and the y-coordinate is the infected percentage of the whole network. Original image from \cite{OnWorms2005survey}}
%\label{SEMmodel}
%\end{figure}

%Figure \ref{SEMmodel} laat het SEM model zien.
%This simple epidemic model has been used in various papers \cite{OwnInternetSI}, \cite{CodeRed} to model random scanning worm such as \textit{Code Red} and \textit{Slammer}. \\
%For our paper we have to approximate the propagation speed en Infect gelijkstellen aan het aantal genoeg nodes die geinfecteerd moeten zijn door de APT om de controle te hebben. Dan weten we hoelang het duurt. In het begin zullen er ook geen geinfecteerde hosts zijn, dus de vergelijking kan opgelost worden.\\
%%Er wordt geen rekening gehouden met de topology ? nog uitzoeken.
To extract the delay from the formula we need the total amount of nodes in the network. If formula \ref{SIdelay} is equal to this amount, variable \textit{t} will be the value of the delay that we need. This is only applicable to nodes in a homogeneous network. A homogeneous network is a network where every node has approximately the same degree. This model is thus suitable for propagation of worms that are topology independent (e.g scan-based worms), because every node has the same chance to be infected by another node in the network. The \textit{Code Red} worm, which is a random scan based worm, has been analysed by this kind of model in \cite{OwnInternetSI}.

\subsection{RCS model}

RCS model stands for Random Constant Spread model and is developed by Paxson and Weaver at Stanford \citep{OwnInternetSI}. It is a model derived from the classical Simple Epidemic Model. They used this model to analyse the \textit{Code Red I} worm. For this model it is assumed that the worm owns a good random number generator that is properly seeded. \\
Let\textit{ N} be the total of vulnerable hosts in the network that can be potentially infected. It is assumed that no system is patched, shut down, deployed or disconnected, which means that the number of hosts in the system stay constant. The model also ignores spreads of the worm behind firewalls on private networks. \\
Let K be the initial compromise rate. This is the rate of hosts that the worm can find and compromise per hour at the beginning of the infection. \textit{K} is a global constant and therefore does not depend on the speed of the network or the processor speed.  Every machine can infect only one other machine after it has been compromised. It cannot increase the rate that a worm can find new hosts. 
The internet topology is considered as a complete undirected graph.  The model type is SI, so the state of the host can only be infected or suspected. Once the host is compromised it stays that way.  
Let \textit{T} be the moment of the start of the infection. Variable \textit{a} is the proportion of vulnerable hosts that has been compromised. Variable \textit{t} is the time in hours. \\

The formula to model the spread of the worm is as follows:
\begin{equation}
N da = (N a)K(1 - a)dt
\end{equation}
It tells how many vulnerable machines will be compromised in the next amount of time \textit{dt}, when it is known the proportion of machines \text{a} that already have been compromised.\\

From this, it follows the simple differential equation:
\begin{equation}
\dfrac{d a}{dt} = Ka(1-a)
\end{equation}
With the following solution:
\begin{equation}
a = \dfrac{e^{K(t-T)}}{1+e^{K(t-T)}}
\end{equation}
\\

Toextract form this formula the delay for the case of FlipIt with a propagation delay, we need to know the maximum numbers of hosts that can be infected before the total system is compromised. The initial compromise rate has to be approximated. \textit{t} is the variable in our formula and says how many time has passed. If \textit{a} passes the proportion of nodes that have to be compromised before the whole system is compromised by the attacker, value \textit{t}  is the value of \textit{d} in our FlipIt game.

%\subsubsection{Kermack-Mckendrick model: SIR}
%In the Kermack-Mckendrick model, also known as SIR model, nodes can have three states: susceptible, infected or removed. Once a node of the network has been recovered from a worm, the node will stay in the removed mode and never becomes infected again. These nodes are not able to infect other nodes and can no longer be infected. \\
%Let I(t) be the number of infectious hosts at time \textit{t}, R(t) be the number of removed hosts at time \textit{t} and J(t) is the number of infected hosts by time \textit{i}, regardless the fact that a node can be in a removed state.
%\begin{equation}
%J(t) = I(t) + R(t).
%\end{equation} 
%The Kermack-McKendrick model can be represented as follows:
%\begin{equation}
%\begin{Bmatrix} \dfrac{d J(t)}{dt} = \beta J(t) \big[N- J(t) \\
%\dfrac{d R(t)}{dt} = \gamma I(t) \\
%J(t) = I(t) + R(t) = N - S(t)
% \end{Bmatrix}
%\end{equation}
%Parameter $\beta$ is again the rate of infection and $\gamma$ is the rate of removal of infected hosts. S(T) is the number of susceptible hosts at time t.
%\begin{figure}[hbtp]
%\centering
%\includegraphics[scale=0.5]{Images/KMmodel.png}
%\caption{KM Model. The x-coordinate is the propagation time and the y-coordinate is the infected percentage of the whole network. Original image from \cite{OnWorms2005survey}. $N=10000, \beta = 1/10000000$.}
%\label{KMmodel}
%\end{figure}
%
%%codered%
%--Define $\rho = \gamma/\beta$ to be the relative removal rate [3]. One
%interesting result coming out of this model is
%dI(t)
%dt > 0 if and only if S(t) > p. (6)
%Since there is no new susceptible host to be generated,
%the number of susceptible hosts S(t) is a monotonically decreasing
%function of time t. If S(t0) < p, then S(t) < p and
%dI(t)/dt < 0 for all future time t>t0. In other words, if the
%initial number of susceptible hosts is smaller than some critical
%value, S(0) < p, there will be no epidemic and outbreak
%[15].
%The Kermack-Mckendrick model improves the classical
%simple epidemic model by considering that some infectious
%hosts either recover or die after some time. However, this
%model is still not suitable for modeling Internet worm propagation.
%First, in the Internet, cleaning, patching, and filtering
%countermeasures against worms will remove both susceptible
%hosts and infectious hosts from circulation, but KermackMckendrick
%model only accounts for the removal of infectious
%hosts. Second, this model assumes the infection rate
%to be constant, which isn't true for a rampantly spreading
%Internet worm such as the Code Red worm--
%
%%\subsubsection{Two-Factor model}
%%%distributed worm simulation
%%--Zou et
%%al. [15] present a ''two-factor'' model that extends SIR
%%epidemiological model to capture the effects of human
%%countermeasures and the congestion due to the worm
%%spread. Shen et al. [16] provide a discrete-time worm
%%model that considers patching and cleaning effect and can
%%model worms with local scanning techniques. All these
%%approaches abstract specifics of the Internet topology,
%%change in the size of vulnerable population as the worm
%%spreads, and the effect of the individual host and network
%%defenses (except for patching, in case of [16]) on the
%%spread--
\subsection*{self disciplinary worms}
Model for self disciplinary worms and counter measures ... []  \todo{verder uitwerken ? ja of nee? }


%Popular mechanism that worms use to detect vulnerable targets by random IP scanning probing. Feasible due to use 32-bit addresses. 128-bit addresses life harder for worms, except the ones that use email systems to propagate. two new strategies: uniformly distributed random number generator to select new target. :spread locally, by biasing the search space towards addresses within the same subnet or network.  
%The second strategy is almost the same as what email virusses would get. For this reason we work an example out .. 
\subsection{Bluetooth worm model}

The Bluetooth worm model is introduced by Yan and Eidenbenz \citep{yan2009modeling}. The model captures the behaviour of the propagation of a worm that spreads through the Bluetooth protocol. 
A Bluetooth device that is compromised can only infect neighbour devices that are in its radio range
Let \textit{i(t)} be the average density of infected hosts in the network at time \textit{t}.

Formula is as follows: 
\begin{equation}
i(t_{k+1})=i(t_{k}) \cdot \dfrac{\rho(t_{k})}{i'(t_{k}) + (\rho(t_{k}) - i'(t_{k}))e^{-\alpha' \cdot \rho(t_{k}) / (\rho(t_{k}) - i'(t_{k}))}}
\end{equation}
where $\rho$(t) and $\beta$(t) are the average device density and the pairwise infection rate at time \textit{t} respectively. The number of new infections out of the infection cycle is denoted by $\alpha$(t).\\

The formula for a better estimation of the worm propagation
\begin{equation}
\alpha'=\dfrac{\rho(t_{k})-t(t_{k})}{\rho(t_{k})} \cdot \alpha(t_{k}) + \dfrac{i(t_{k})}{\rho(t_{k})} \cdot \alpha(t_{x})
\end{equation}

To apply this model on FlipIt we again set up a threshold value for i(t) and see how long it took to compromise this amount of hosts.  The paper concluded in their work  that after setting model parameters accordingly ($\lambda_{ne} = 0.2108$ the average node degree and $J_{in}= 0.2372$ the average meeting rate of neighbours), the model predicts that the time it would take to infect 99$\%$ of the devices is slightly less than one hour. So in this case the delay is equal to an hour and the amount of hosts that has to be infected by the attacker before the network is compromised is 99$\%$ of the hosts in the network.

%\subsubsection{OSN worm model}
%OSN worms stands for Online Social Network worms. This model is based on worms that propagate through the use of social media, like Facebook. The difference between an email worm and a social network worms is that the topology of the network is different. 

\subsection{AAWP model}
SIR AAWP stands for Analytic Active Worm Propagation. The AAWP model is a SIR model which uses a death rate in the formula to calculate the amount of hosts that become Removed. Yet this model is still analysed because if the death rate is removed, this model becomes a SI mode. The model is different from other SI models because it includes the time that it takes to infect a host. A hosts cannot infect another hosts before it is completely compromised. The model also considers the fact that a hosts can be scanned and hit by multiple worms at the same time. %This model is used for worms that propagate through uniform scanning. When uniform scanning is used, each IP address 

The spread of the AAWP model is characterized as follows:
\begin{equation}
I_{t+1}=I_{t}+(N-I_{t})[1-(1-\dfrac{1}{\Omega})^{sI_{t}}]
\end{equation}
where $I_{t}$ is the amount of infected hosts at time \textit{t}, \textit{N} is the number of vulnerable hosts, \textit{s} is
the scanning rate of the worm and $\Omega$ is the scanning
space.


%Popular mechanism that worms use to detect vulnerable targets by random IP scanning probing. Feasible due to use 32-bit addresses. 128-bit addresses life harder for worms, except the ones that use email systems to propagate. two new strategies: uniformly distributed random number generator to select new target. :spread locally, by biasing the search space towards addresses within the same subnet or network.  
%The second strategy is almost the same as what email virusses would get. For this reason we work an example out .. \\



%A method to calculate the propagation of the virus in an easy way. Google page ranking algorithm. 


\section{Matrix worm model}
\label{eigenmatrixmethode}
As can be seen in table \ref{tree} most propagation models are for a specific network topology. The different kinds of network topology of each system is either homogeneous, small-world network, random network or power-law network. A homogeneous network is a network where every node has about the same degree of connectivity. This means that every node can infect each node with the same opportunity. A small-world topology is a network where most nodes are not connected with each other but they can all reach each other in a few steps. Social networks are an example of a small-world network. A random network is a topology where each connection is chosen at random with equal probability. In a power law network the degrees of each node in the network follow the power law. That is that some of the nodes have a small degree and other have a very large degree of connection. What we want to have is to have a method to calculate the delay that is independent of the network topology. A method where we can chose in the beginning which topology we have and then calculate the delay.\\

%All of the above models for modelling the mechanism of the spreading of a worm depend on the kind of topology of the network.  
In this chapter we propose a method to calculate the spread of the worm in way that the topology of the network can be easily integrated. The method will give an approximation of how fast a worm can infect a network. The method is based on the method for Google Page Ranking \cite{GoogleRank}. 

\subsubsection{Google PageRank algorithm}
The PageRank algorithm is introduced by Page and Brin in 1998 as one of the main features of the search engine Google to improve the search results. PageRank models the human behaviour when users surf through the net. It can also model \textit{random surfers}. A random surfer is the probability that a surfer gets bored and randomly visits another page which is not linked with the initial page. The probability that he will visit a random page is the PageRank of that page. A page will have a high PageRank if many pages will point to this page. This means that this page is well cited through other pages and maybe an important one to look at.    The main idea of the PageRank algorithm is to look at the number of (important) web pages that point to a particular page. The ranking of this page depends than on the number of out coming links and the importance of the pages that link to this page. With a probability of $P$ the surfer will follow a link of the page to another page, and with a probability of $1-P$ the surfer will surf to a random page. The PageRank algorithm is expressed as follows:
\begin{equation}
PageRank(A)=P \sum \dfrac{PageRank(i)}{d_{out,i}} + (1-P) \cdot e_{A}
\end{equation}
where $N_{A}$ is the number of pages, $PR(i)$ is the ranking of web page $i$, $D_{out,i}$ is the number of outgoing links of page $i$, $(1-P)$ the probability that a random page is searched, and $e_{A}$ the restart value for web page A which is often uniformly distributed among all web pages.

\subsubsection{Model with matrix}


For some of the spreading methods, the graph of the network matters. It is important to have the right topology for the right method. Email worms need a topology that represent a social network, BGP routing worms need a topology on network level. \\
%--iets zeggen dat het ook belangrijk is voor de defender om zijn netwerk zo aan te passen dat het virus moeilijker kan verspreiden. Network segregation mss aanhalen ?--
%
%\begin{description}
%\item Power law
%\item Small world topology
%\item random graph
%\item ..
%\end{description}

%The usefulness of this method is that it is graph model independent. The topology of the network has to be in the form of a square matrix. Each connection is determined by the entries on the matrix. 
%$N_{0}$ denotes the initially infected resources at the beginning of the virus propagation. 
%$P_{x}(R_{n},t|R_{0},r,t_{0})$ denotes the chance that resource $R_{n}$ is infected at time $t$ after dropping virus number x on to resource $R_{0}$ at $t_{0}$ with rate $r$. \\

%$S(R_{n},R_{0})$ denotes the shortest path from the infected resource $R_{0}$ to resource $R_{n}$. It gives back a value with the distance measured with how many resources are in between including the end resource. \\

%So the chance that a resource is infected after time t is the chance that the resource is infected by all the previous infections and that the defender has not flipped the resource.

The computer network can be modelled by an undirected Graph $G = < V, E> $ where $|V|$ denotes the number of resources in the network and $|E|$ the number of connections. We can convert this to an adjacency matrix which represents which vertices of the graph are neighbours of other vertices. \\
The graph is represented as a $|V| \times |V|$ matrix with for every entry $a_{ij}$ a 1 as value if there is a connection between node $V_{i}$ and $V_{j}$ and 0 otherwise, and with 0's for every $a_{ii}$. Because the graph is undirected we have a symmetric matrix.  \\ 

Adjacency matrices have many interesting applications, amongst which calculating the paths between vertices:
\textit{``If \textit{A} is the adjacency matrix of the directed or undirected graph \textit{G}, then the matrix $A^{n}$ (i.e., the matrix product of \textit{n} copies of \textit{A}) has an interesting interpretation: the entry in row \textit{i} and column \textit{j} gives the number of (directed or undirected) walks of length \textit{n} from vertex \textit{i} to vertex \textit{j}. If \textit{n} is the smallest nonnegative integer, such that for all i ,j , the (i,j)-entry of $A^{n} > 0$, then n is the distance between vertex i and vertex \textit{j}.''}  source: \cite{wikimatrix} .\\


%%% Local Variables: 
%%% mode: latex
%%% TeX-master: "thesis"
%%% End: 
\begin{figure}
\centering
\begin{tikzpicture}[->,>=stealth',shorten >=1pt,auto,node distance=2.8cm,
                    semithick]
  \tikzstyle{every state}=[fill=black!60!green,draw=none,text=white]

  \node[initial,state] (A)                    {$N_1$};
  \node[state]         (B) [above right of=A] {$N_2$};
  \node[state]         (D) [below right of=A] {$N_3$};
  \node[state]         (C) [below right of=B] {$N_4$};
  \node[state]         (E) [below right of=C] {$N_5$};
  \node[state]		   (F) [above right of=C] {$N_6$};

  \path (A) edge              node {} (B)
            edge              node {} (D)
        (B) edge              node {} (A)
        	edge			  node {} (C)
        (C) edge              node {} (B)
            edge 			  node {} (D)
            edge			  node {} (E)
            edge			  node {} (F)
        (D) edge 			  node {} (C)
            edge              node {} (A)
        (E) edge 			  node {} (C)
    	(F)	edge			  node {} (C);
\end{tikzpicture}
\caption{Network with 6 nodes. The arrows represent the connections between the nodes. The start point is were the worm has been dropped, here node 1.}
\label{netwerkfiguur}
\end{figure}

%\[
%\begin{bmatrix}
%    x_{11}       & x_{12} & x_{13} & \dots & x_{1n} \\
%    x_{21}       & x_{22} & x_{23} & \dots & x_{2n} \\
%    \hdotsfor{5} \\
%    x_{d1}       & x_{d2} & x_{d3} & \dots & x_{dn}
%\end{bmatrix}
%=
%\begin{bmatrix}
%    x_{11} & x_{12} & x_{13} & \dots  & x_{1n} \\
%    x_{21} & x_{22} & x_{23} & \dots  & x_{2n} \\
%    \vdots & \vdots & \vdots & \ddots & \vdots \\
%    x_{d1} & x_{d2} & x_{d3} & \dots  & x_{dn}
%\end{bmatrix}
%\] 
Assuming a network like in figure \ref{netwerkfiguur}, the corresponding adjacency matrix is the  matrix $[A]$: \\


$
\bordermatrix{
         & N_1		& N_2	& N_3	& N_4 	& N_5	&N_6     \cr
    N_1   & 0		& 1		& 1		& 0		& 0		& 0	     \cr
    N_2   & 1		& 0		& 0		& 1		& 0		& 0	     \cr
    N_3   & 1		& 0		& 0		& 1		& 0		& 0	     \cr
    N_4   & 0		& 1		& 1		& 0		& 1		& 1	     \cr
	N_5   & 0		& 0		& 0		& 1		& 0		& 0	     \cr
	N_6   & 0		& 0		& 0		& 1		& 0		& 0	     \cr
}$
\\

In matrix $A \times A = A^{2}$, each entry represents the number of 2 step paths from $N_{i}$ to $N_{j}$: We denote this matrix as matrix \textit{[B]}\\


$
\bordermatrix{
         & N_1		& N_2	& N_3	& N_4 	& N_5	&N_6     \cr
    N_1   & 2		& 0		& 0		& 2		& 0		& 0	     \cr
    N_2   & 0		& 2		& 2		& 0		& 1		& 1	     \cr
    N_3   & 0		& 2		& 2		& 0		& 1		& 1	     \cr
    N_4   & 2		& 0		& 0		& 4		& 0		& 0	     \cr
	N_5   & 0		& 1		& 1		& 0		& 1		& 1	     \cr
	N_6   & 0		& 1		& 1		& 0		& 1		& 1	     \cr
}$
\\

Likewise, in matrix $A^{2} \times A = A^{3}$ each entry represents the number of paths with 3 steps from $N_{i}$ to $N_{j}$: We denote this matrix as matrix \textit{[C]}\\


$
\bordermatrix{
         & N_1		& N_2	& N_3	& N_4 	& N_5	&N_6     \cr
    N_1   & 0		& 4		& 4		& 0		& 2		& 2	     \cr
    N_2   & 4		& 0		& 0		& 6		& 0		& 0	     \cr
    N_3   & 4		& 0		& 0		& 6		& 0		& 0	     \cr
    N_4   & 0		& 6		& 6		& 0		& 4		& 4	     \cr
	N_5   & 2		& 0		& 0		& 4		& 0		& 0	     \cr
	N_6   & 2		& 0		& 0		& 4		& 0		& 0	     \cr
}$ 


~~\\
So, in $A^{N}$ every $a_{ij}$ entry gives the number of paths with N steps from $N_{i}$ to $N_{j}$.\\

Using this knowledge we can calculate in how many steps a node is infected: \\
The infection vector or start vector $I_{0}$ of lenght $|V|$ indicates which node is infected and which not. Multiplying the infection vector with $A$ results in a vector $I_{1}$, which represents which nodes are infected after 1 step. Multiplying the start vector $I_{0}$ with $A^{N}$ calculates which nodes are infected in \textit{N} steps: each non zero entry represents an node infected after \textit{N} steps. \\
In the context of the FlipIt game, it is safe to assume that once a node is infected, it stays infected until the defender Flips the node. This means that after d steps, it is assumed that all nodes that could be reached in less than \textit{d} steps from the node where the worm was dropped, are infected. In order to know how many nodes are infected after (for example) at most 3 steps, we have to consider nodes that are infected initially (step 1), or after 2 steps, or after 3 steps.  Calculating the sum of the three matrices $(A + A^{2} + A^{3}) $ results in a matrix that indicates for each node, the number of paths of length 1, 2 or 3 from \textit{i} to \textit{ j}. Multiplying the start vector with this matrix, results in a vector that indicates which nodes will be infected in at most 3 steps . This technique can be applied to calculate the state of the network after any number of steps. \\

Determining the length of the delay boils down to determining which configuration of infected nodes is considered as corresponding to the attacker having flipped the resource. It may be that all nodes need to be flipped, a sufficient amount of nodes, or a (set of) particular nodes.\\

--Hier nog een beetje verder uitschrijven hoe je exact de d dan moet bepalen, gegeven dat je niet weet in welke knoop het virus gedropt wordt-- \\


%What do we need for an algorithm
%\begin{description}
%\item Graph network $G = < V, E>$
%\item Graph matrix $[A]$ which is $|V| \times |V| $
%\item Attack vector $[X]$ which is $1 \times |V|$
%\item cummulative matrix $[M]$ which is $|V| \times |V|$
%\item state matrix $[T]$  which is $|V| \times |V|$
%\item Reset vector $[R]$
%\item duration \textit{d}
%\item time \textit{n}
%\item rate $\delta _{0}$ of defender and $\delta _{1}$ of attacker
%\end{description}
%
%
%
%Initialisation algorithm:
%
%
%\begin{verbatim}
%initialisatie
%	d=0
%	A=basismatrix
%	M=A^{0}
%	n=0
%	\delta_{0}
%	\delta_{1}
%	X
%	R
%	controller = defender
%	
%	
%
%	Algorithm
%	n:= n + 1;
%	Check who is in control? ( through modulo )
%	if ( defender & controller=defender)
%				d:= d + 1;
%	
%	if ( defender & controller=attacker )
%				G = X \times R  (flippen ten voordele van defender)
%				d = 0
%				controller = defender
%				
%	if ( attacker & controller=defender )
%				controller=attacker
%				..
%				
%	if ( attacker & contoller=attacker )
%				d:= d + 1
%				M = M x A
%				T = T + M
%				G = X x T
%				
%		
%\end{verbatim}


%\end{document}
